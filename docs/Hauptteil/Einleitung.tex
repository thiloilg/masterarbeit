\chapter{Einleitung}
\label{chap:einleitung}
Visualisierungsanwendungen bieten dem Benutzer die Möglichkeit,
Daten in Wissen umzuwandeln. Die durch die Anwendung bereitgestellten
Daten können auf explorative Weise analysiert werden, um so neue
Erkenntnisse zu gewinnen. Oftmals hat die Zeit, die zwischen der 
Datenerhebung und der aus der Datenvisualisierung entstandenen Erkenntnis
hervorgeht, Einfluss auf die Effizienz der daraus resultierenden Entscheidungen.
Handelt es sich bei den zu visualisierenden Daten um Maschinendaten,
können mithilfe einer zeitnahen Analyse Störfälle verhindert werden.

Die Masterarbeit findet in Zusammearbeit mit GBO Datacomp statt.
Das mittelständische Unternehmen bietet Lösungen für die Erfassung,
Weiterverarbeitung und Auswertung von Betriebsdaten an. Die Daten
beinhalten Informationen über die Aufträge, Schichten und Maschinen
des Betriebes. GBO Datacomp stellt Beispieldaten in Form einer REST-API
zur Verfügung. Die Aufgabe der Masterarbeit ist es, ein System zu entwickeln, das die 
bereitgestellten Daten abfragt, verarbeitet und in einem webbasierten Dashboard
darstellt.

Die Technologien des Internets befinden sich in einem stetigen Entwicklungsprozess.
Mit einer großen Anzahl moderner APIs ist es möglich, Webanwendungen Funktionalitäten
hinzuzufügen, die bis vor kurzem nativen Apps vorbehalten waren. So können Entwickler
Apps für mobile Endgeräte, Webseiten und Desktopanwendungen aus der gleichen Codebasis
heraus entwickeln.

Auch im Bereich der stetigen Auslieferung der Software sowie dem Ausführen des Gesamtsystems in der 
Produktion hat sich einiges getan. So kann ein einzelner Entwickler von zu Hause aus über einen automatisierten 
Auslieferungsprozess ein komplettes Servercluster aufziehen. Dieses kann je nach Anzahl aktiver Nutzer skaliert
und ohne Ausfallzeit aktualisiert werden.

Ziel der Arbeit ist es, neue Technologien und 

wie progressive Webanwendungen und 


% motivation
 
% zielsetzung

% gliederung


% \Cref{chap:konzept}
% bei der Einleitung werden die einzelnen Kapitel mit Cref Referenziert!!!

% Überblick über die Struktur der Kapitel

% mindestens 5 Seiten!

% Masterarbeit in Zusammenarbeit mit GBO Datacomp

% Open Source Projekt, soll ein Anreiz sein, was möglich ist
