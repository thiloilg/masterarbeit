\chapter{Einleitung}
\label{chap:einleitung}
Visualisierungsanwendungen bieten dem Benutzer die Möglichkeit,
Daten in Wissen umzuwandeln. Die durch die Anwendung bereitgestellten
Daten können auf explorative Weise analysiert werden, um so neue
Erkenntnisse zu gewinnen. Oftmals hat die Zeit, die zwischen der 
Datenerhebung und der aus der Datenvisualisierung entstandenen Erkenntnis
hervorgeht, Einfluss auf die Effizienz der daraus resultierenden Entscheidungen.
Handelt es sich bei den zu visualisierenden Daten um Maschinendaten,
können mithilfe einer zeitnahen Analyse Störfälle verhindert werden.

Die Masterarbeit findet in Zusammearbeit mit GBO Datacomp statt.
Das mittelständische Unternehmen bietet Lösungen für die Erfassung,
Weiterverarbeitung und Auswertung von Betriebsdaten an. 



\section{Motivation}
\label{sec:motivation}


\section{Zielsetzung}
\label{sec:zielsetzung}

\section{Aufbau der Arbeit}
\label{sec:aufbauderarbeit}
% \Cref{chap:konzept}
% bei der Einleitung werden die einzelnen Kapitel mit Cref Referenziert!!!

% Überblick über die Struktur der Kapitel

% mindestens 5 Seiten!

% Masterarbeit in Zusammenarbeit mit GBO Datacomp

% Open Source Projekt, soll ein Anreiz sein, was möglich ist
