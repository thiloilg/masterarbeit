\chapter{Einleitung}
\label{chap:einleitung}
Visualisierungsanwendungen bieten dem Benutzer die Möglichkeit,
Daten in Wissen umzuwandeln. Die durch die Anwendung bereitgestellten
Daten können auf explorative Weise analysiert werden, um so neue
Erkenntnisse zu gewinnen. Oftmals hat die Zeit, die zwischen der 
Datenerhebung und der Erkenntnis, die aus den visualisierten Daten hervorgeht,
Einfluss auf die Effizienz der daraus resultierenden Entscheidungen.
Handelt es sich bei den zu visualisierenden Daten um Maschinendaten,
können mithilfe einer zeitnahen Analyse Störfälle verhindert werden.

Die Technologien des Internets befinden sich in einem stetigen Entwicklungsprozess.
Mit einer großen Anzahl moderner Web-Technologien ist es möglich, Webanwendungen Funktionalitäten
hinzuzufügen, die bis vor kurzem nativen Apps vorbehalten waren. So können Entwickler
Apps für mobile Endgeräte, Webseiten und Desktopanwendungen aus der gleichen Codebasis
heraus entwickeln.

Auch im Bereich der stetigen Auslieferung der Software sowie dem Ausführen des Gesamtsystems in der 
Produktion hat sich einiges getan. So kann ein einzelner Entwickler von zu Hause aus über einen automatisierten 
Auslieferungsprozess ein komplettes Servercluster aufziehen. Dieses kann je nach Anzahl aktiver Nutzer skaliert
und ohne Ausfallzeit aktualisiert werden.

Die Masterarbeit findet in Zusammearbeit mit GBO Datacomp statt.
Das mittelständische Unternehmen bietet Lösungen für die Erfassung,
Weiterverarbeitung und Auswertung von Betriebsdaten an. Die Daten
beinhalten Informationen über die Aufträge, Schichten und Maschinen
des Betriebes. GBO Datacomp stellt Beispieldaten in Form einer REST-API
zur Verfügung. Die Aufgabe der Masterarbeit ist es, ein System zu entwickeln,
das die bereitgestellten Daten abfragt, verarbeitet und in einem webbasierten
Dashboard darstellt. 

Zur Entwicklung des Gesamtsystems sollen moderne Web-Technologien verwendet
und ein automatisierter Auslieferungsprozess implementiert werden. Ziel der Arbeit ist es,
Tests bereits vor der Implementierung des eigentlichen Quellcodes zu schreiben. Des Weiteren 
bedarf es einer Automatisierung dieser als Teil des Auslieferungsprozesses. 

Um eine langlebige, flexible Softwarearchitektur zu erreichen, gilt es die Logik
der Gesamtanwendung in einzelne Services zu unterteilt. Ein Teilziel der Arbeit
ist es, dass die einzelnen Services horizontal skalierbar sind. Mehrere Instanzen
der Services können hochgefahren werden, ohne das die Zusammenarbeit
des Gesamtsystems beeinträchtigt wird. Dies soll dazu dienen, dass die Anwendung
auch in einer größeren Firma mit mehreren tausend Mitarbeitern verwendet werden kann.

Die Masterarbeit ist in neun Kapitel unterteilt. In \Cref{chap:grundlagen} geht es um für
die Masterarbeit relevantes Basiswissen. Als erstes werden die Merkmale einer progressiven
Webanwendung vorgestellt. In einer darauf folgenden Untersuchung wird der
unterstützte Funktionsumfang gängiger Browser analysiert und ausgewertet.
Außerdem beschäftigt sich das Kapitel der Grundlagen mit Microservices, einem serviceorientierten
Architekturmuster, sowie dem stetigen Auslieferungsprozess einer Softwareanwendung.

Die Analyse bestehender Softwarelösungen wird in \Cref{chap:analysebestehendersoftwareloesungen}
angegangen. Hier werden die Softwarelösungen Power BI, Qlik Sense und Tableau anhand von für
die Arbeit relevanter Kriterien begutachtet.

In \Cref{chap:anforderungsanalyse} werden die Anforderungen an die zu implementierende Software
gestellt. Anfänglich wird die Zielgruppe betrachtet und daraus bedeutsame Erfordernisse
für die Gesamtanwendung erarbeitet. Überdies werden funktionale und nicht-funktionale Anforderungen
an die Softwarelösung diskutiert. Die nicht-funktionalen Anforderungen sind in 
Gestaltung, Flexibilität und Codebasis unterteilt.

Die Gliederung von \Cref{chap:konzept} orientiert sich an dem Datenstrom der zu visualisierenden
Betriebsdaten. Das Kapitel behandelt diverse Konzepte zur performanten Echtzeitauslieferung, Verarbeitung
und Veranschaulichung und Speicherung der Daten.

In \Cref{chap:implementierung} wird die Implementierung der Anwendung dargestellt. Aufgrund der Komplexität
des Gesamtsystems ist es das längste Kapitel. Das Kapitel geht von der Infrastruktur über die Designentscheidungen
der Komponenten des Systems bis hin zu Implementierungsdetails der einzelnen Services.

Die Auswertung der Resultate findet in \Cref{chap:auswertung} statt. Hier werden die in \Cref{chap:anforderungsanalyse}
gestellten Anforderungen mit den Ergebnissen der Implementierung verglichen.

Das Fazit und der Ausblick werden in \Cref{chap:fazit} und \Cref{chap:ausblick} behandelt. Im Anhang befinden
sich Screenshots der Frontendanwendung, weitere Quellcodeausschnitte sowie Informationen
über den beigefügten Datenträger.
