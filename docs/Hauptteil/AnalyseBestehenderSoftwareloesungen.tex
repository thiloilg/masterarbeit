\chapter{Analyse bestehender Softwarelösungen}
\label{chap:analysebestehendersoftwareloesungen}

Um einen besseren Überblick über Softwarelösungen im Bereich der Geschäftsanalytik zu erhalten, beschäftigt sich
die Arbeit im folgenden Kapitel mit drei etablierten Softwarelösungen; Power BI von
Microsoft, Qlik Sense von QlikTech und Tableau von Tableau Software. Alle diese drei Anwendungen sind Werkzeuge,
die dabei helfen, Wissen aus Daten zu entnehmen. Bei der Auswertung der Softwarelösungen soll der Fokus auf dem Laden
von Daten aus Datenquellen, der Aufbereitung der Daten, der Datenmodellierung sowie der Erstellung von Dashboards liegen.

\section{Power BI Desktop}
\label{sec:powerbidesktop}

Microsoft beschreibt ihr Produkt Power BI Desktop wie folgt: "Power BI Desktop ist eine kostenlose Anwendung,
die Sie auf Ihrem lokalen Computer installieren können und die es Ihnen ermöglicht, eine Verbindung mit Ihren
Daten herzustellen und diese zu transformieren und zu visualisieren."\cite{MicrosoftPowerBIDesktopDocs}
Power BI Desktop ist eines der Produkte von Power BI. Außerdem bietet Power BI einen cloud-basierten Dienst
sowie eine native App an.\cite{WikiPowerBI}

Als erstes gilt es, das Laden von Daten in Power BI Desktop auszuwerten. Mit Power BI Desktop können
Daten aus Dateien, aus Datenbanken sowie aus cloud-basierten Diensten geladen werden. In dem Buch "Pro Power BI Desktop"
von Adam Aspin wird eine weitere Option zum Laden von Daten erwähnt; die DirektQuery.
Hierbei handelt es sich, anders als bei den zuvor genannten Möglichkeiten, um das direkte
Abfragen von Daten gegen eine Datenquelle, ohne zuvor alle Daten in die Anwendung zu laden.\cite[S. 111]{ProPowerBIDesktop}
Diese Art des Ladens wurde von den Entwicklern von Power BI unter anderem eingeschlagen,
um das initiale Laden der Daten in das In-Memory-Model zu umgehen. Gerade bei großen Data Warehouses
kann sich dies mitunter unvorteilhaft auf die Performanz auswirken.

Eine weitere für die Arbeit relevante Frage ist, wie komplexe Datenstrukturen, wie multidimensionale JSON-Dateien,
in die anwendungseigene Datenstruktur übersiedelt werden. Hierfür bietet der Power BI Kosmos eine extra Software
namens Power Query an. Es wäre falsch mit hundertprozentiger Sicherheit zu behaupten, dass Power BI in der
anwendungseigenen Datenstruktur ein relationales Datenbankmodell verwendet. Befasst man sich mit der Art
der Datenmodellierung, wie der Erstellung von Beziehungen zwischen Tabellen, verschwindet jedoch jedweder Zweifel.\cite[S. 319]{ProPowerBIDesktop}
Somit ist davon auszugehen, dass Power Query die zu importierenden Datenstrukturen in ein relationales Datenbankmodell
konvertiert. Eine mehrdimensionale JSON-Datei wird somit in eine eindimensionale Tabelle
konvertiert.\footnote{Die Tabelle beinhaltet die weiteren Dimensionen in den jeweiligen Reihen als Arrays.
Die Tabelle kann daraufhin in mehrere Tabellen mit den jeweiligen Kardinalitäten konvertiert werden.}
Hierfür benötigt man die hauseigene Formelsprache Power Query M. "Power Query M ist eine funktionale Sprache,
die Groß-/Kleinschreibung beachtet und F\# ähnlich ist."\cite{MicrosoftDocsPowerQueryFormelsprache}

Eine interessante, neue Funktionalität im cloud-basierten Power BI Dienst ist die Self-Service-Datenaufbereitung,
bekannt als Dataflow. Mit Dataflows können ETL-Prozesse\footnote{ETL steht für das Extrahieren, Transformieren und Laden von Daten}
unkompliziert innerhalb des Power BI Kosmos bewerkstelligt werden. Ein automatisierter Datenfluss,
der die Daten in eine für die Visualisierung einheitliche Struktur vorbereitet, wird auch in dieser
Arbeit notwendig sein.

Bei der Erstellung eines Dashboards in Power BI, zieht man zuerst ein Diagrammtyp wie ein Kuchendiagram
auf den weißen Untergrund. Als zweiten Schritt weist man ebenfalls per Drag and Drop dem Kuchendiagram ein
Datenfeld zu. Fährt man mit der Maus an die Ränder eines Diagramms, kann dieses in eine beliebige Größe
gezogen werden. Alles in allem ist das Erstellen von Dashboards in Power BI Desktop ein sehr intuitives,
benutzerfreundliches Verfahren.

\section{Qlik Sense}
\label{sec:qliksense}
Als nächstes werfen wir einen Blick auf die Softwarelösung Qlik Sense.
Das Laden von Daten ist bei Qlik Sense von Dateien, Datenbanken, Webseiten, FTP-Servern sowie von einer
REST Schnittstelle aus möglich. Bei dem Laden der Daten von einer Webseite, wird die Seite nach einer Tabelle
durchsucht und diese als Datenquelle verwendet.\cite[S. 17]{QlikSenseCookbook} Für das Laden von Daten von
einer REST-API, bietet Qlik Sense einen sogenannten REST Connector an. Dieser kann sowohl das \mbox{JSON-,} als auch
das XML-Format verarbeiten. In den Settings des REST Connectors können neben der URL auch ein
Authentifizierungsschema ausgewählt werden.\cite[S. 23]{QlikSenseCookbook} Somit ist es möglich,
eine API zu verwenden, bei der eine JWT-Authentifizierung vorausgesetzt ist. Qlik Sense besitzt
eine eigene SQL ähnliche Skriptsprache. Im Data Load Editor
können mithilfe dieses Scripts Bearer-Token als Authorization Header gesetzt werden. Dies macht Qlik
Sense in der Nutzung externer APIs sehr flexibel.

Als nächstes wenden wir uns der Aufbereitung der Daten zu. Nach der Auswahl einer Datenquelle
erstellt Qlik Sense ein vorgefertigtes Script im Data Load Editor. Dies gilt primär als Orientierungspunkt
und kann je nach Belieben angepasst werden. Genauso wie Power BI benutzt auch Qlik Sense ein relationales
Datenbankmodell zur Datenmodellierung innerhalb der Anwendung. Im Data Load Editor
wird, so wie bei Power BI, das eingehende JSON mithilfe eines Scripts in eine Tabelle konvertiert.
Der Bereich zur Datenmodellierung ist der Data Model Viewer.
Hier werden mithilfe eines kraftgerichteten Graphs die Beziehungen zwischen den unterschiedlichen
Tabellen verdeutlicht. Mit der Datenmodellierung von Qlik Sense ist es möglich, alle typischen SQL-Joins
zwischen zwei Tabellen durchzuführen. So können Daten von einer API zusammen mit Daten aus einer SQL-Datenbank
verbunden werden.

Anders als bei Power BI, das von Microsoft selbst gehostet wird, kann man Qlik Sense nach dem Erwerben
einer Lizenz selbst auf einem Windows Server hosten. Außerdem hat Qlik verschiedene APIs, die man
verwenden kann, um die Anwendung selbst ohne die Verwendung der GUI vorzubereiten. So kann man Daten
auf den Server vorladen, die man später mithilfe der Frontendanwendung auswerten wird. Anders als bei
dem API Connector von Qlik Sense, den man über die GUI einstellt, kann man mit der Repository Main API
die Anwendung komplett über eine Schnittstelle aus verwalten.\cite{QlikSenseRepositoryMainAPI}
Das vereinfacht das Einbinden von Qlik Sense in eine bereits existierende IoT-Umgebung.

Bei der Erstellung von Charts ähnelt Qlik dem Vorgehen von Power BI Desktop. Mit einem intuitiven
Drag and Drop Verfahren können die Benutzer zuerst Diagrammtypen auf das Dashboard ziehen und
daraufhin Daten zuweisen. Bei der Datenzuweisung unterscheidet Qlik Sense zwischen Measures und
Dimensions. Measures sind numerische Werte wie die Anzahl der verkauften Produkte oder der Jahresumsatz,
Dimensions sind deskriptive Werte wie Monat, Jahr oder ProduktID.\cite{TutorialsSpotMeasuresDimensions}
Durch die klare Trennung in numerische sowie descriptive Werte wird die Zuordnung der Daten
auf die verschiedenen Achsen der Diagramme vereinfacht.

\section{Tableau}
\label{sec:tableau}
In diesem Abschnitt gilt es Tableau genauer zu betrachten. Tableau besitzt, wie auch Power BI und Qlik,
eine Infrastruktur an unterschiedlichen Diensten. So gibt es Tableau Desktop, das Equivalent zu Power BI
Desktop, einen Tableau Server sowie native Apps. Bei der Datenbeschaffung bietet Tableau Desktop wie auch
Power BI und Qlik Sense, neben dem Laden von lokalen Dateien und diversen spezifischen Datenbankschnittstellen,
das Laden aus Datenbanken mit einer gängigen ODBC-Schnittstelle an. Daten können in Echtzeit
über SQL-Statements abgefragt werden. Alternativ werden die Daten auf dem Betriebssystem, auf dem
Tableau Desktop verwendet wird, persistiert und mithilfe der eigenen Data Engine agil aus dem
vorgeladenen Arbeitsspeicher geladen.\cite[S. 50]{ProTableauBook}

Tableau besitzt für die Aufbereitung der Daten Tableau Prep ein Werkzeug zur Durchführung der
ETL-Prozesse.\footnote{https://www.tableau.com/de-de/products/prep} Unübersehbar positiv ist
in Tableau Prep das agile Vorgehen zur Erstellung eines Flows. Ein Flow in Tableau Prep ist ein Programmablaufplan
zur Verarbeitung eines Datenflusses. In einem Flow lassen sich per Drag and Drop Datenquellen sowie diverse Verarbeitungsmechanismen zu einem Programmablaufplan
zusammensetzen. Verarbeitungsmechanismen sind SQL-Joins, verschiedene Mechanismen der Datenbereinigung,
das Aufteilen und Zusammenfügen von Feldern sowie das Drehen von Tabellen anhand von definierbaren Drehpunkten.
Die einzelnen Verarbeitungsmechanismen werden in dem Buch "Prepare Your Data for Tableau" von Tim Costello und
Lori Blackshear genauer beschrieben.\cite{PrepareYourDataForTableauBook} Die Flows von Tableau Prep können,
ähnlich wie die Dataflows von Power BI, gespeichert und unter Mitarbeitern geteilt werden. Als Datenstruktur
scheint ein Flow ein gerichteter Graph zu sein, durch den Daten von einer zur anderen Seite durchfließen
können. Es ist allerdings kein Wurzelgraph, da die Flows neben mehreren Datenquellen auch mehrere Datenausgaben
haben können. Die Knoten des Graphen stellen die Verarbeitungsmechanismen dar. Die Darstellung ist sehr statisch
aufgebaut. Es handelt sich hier also nicht wie bei der Datenmodellierung von Qlik um einen kraftgerichteter Graphen.
Fährt man mit dem Mauszeiger über die Knoten des Graphen, lassen sich neue Knoten durch das Klicken auf ein
auftauchendes Plus hinzufügen.

Das Erstellen von Dashboards ist, wie auch bei Power BI und Qlik Sense, durch ein Drag and Drop Verfahren möglich.
Die Daten sind wie bei Qlik Sense in Measures und Dimensions unterteilt. Bei der Platzierung der Diagramme auf dem
Dashboard sind zwei Optionen möglich. Es gibt eine freie sowie eine an einem Raster fixierte Anordnung. Bei der fixierten
Anordnung lassen sich die bereits auf dem Dashboard existierenden Diagramme vertikal sowie horizontal von dem neu 
hinzugefügten Diagramm unterteilen.
