\chapter{Der Durchstich}

\section{DevOps ist dein bester Freund}

\section{Die Unabhängigkeit gegenüber einzelner Unternehmen}
\subsection{Der eigene Gitlab Server}

\section{Wie richte ich meine Pipeline ein?}

\section{Der richtige Umgang mit Umgebungsvariablen}

\section{Das besähen der Datenbank}
%% Database seeding admin account etc.
%% Probleme with extra route to initialize administration account!!!
%% Problem initializing with trivial password since a lot of provider forget to update
%% also known as Populating a Database
%% Problem Typeorm auto sync
%% Neccessary also for as fixtures, a basic setup for running tests

\section{Die Gesundheit der Services}
In Microservice-Infrastrukturen sind die einzelnen Services eng miteinander verküpft
und voneinander abhängig. Um die Verfügbarkeit sicherzustellen, werden auf
Serviceebene Healthcheckendpunkte bereitgestellt. Diese spiegeln den aktuellen
Gesundheitszustand des jeweiligen Dienstes wieder. Ein Healthcheckendpunkt
ist speziell dann nützlich, wenn ein Dienst oder ein Verfahren von einem anderen
abhängig ist. So ist beispielsweise die Backendapi von der Datenbank
abhängig. Andererseits ist der Frontenddiesnt sowie das Apitestverfahren
in der Buildpipeline von der Backendapi abhängig. 

\begin{listing}
    \label{lst:healthcheck}
    \inputminted{sh}{snippets/sh/healthcheck.sh}
    \caption{Healthcheckbeispiel in Docker}
\end{listing}

\section{Die Unabhängigkeit des Betriebssystems}
\subsection{Multistage Builds}

\section{Wer automatisiert, gewinnt auf lange Zeit}
