\chapter{Konzeption}

Zur Erstellung des Konzepts gilt es folgende Aspekte im voraus genauer zu betrachten.
Wer ist meine Zielgruppe? Was ist für die zukünftigen Benutzer der Software wichtig?
Was ist die Anforderungen an die Software? Gibt es bestimmte Abläufe, die von der
Software bewerkstelligt werden müssen? Wie flexibel muss die Software auf Änderungswünsche
reagieren können? Was ist bei der Gestaltung der Benutzeroberfläche wichtig? 
Welcher Tech-Stack wird den Anforderungen gerecht? Was für Daten werden die Kunden mit der
Software auswerten? Wie und in welcher Form sollten die Daten dargestellt werden? 
Fragen über fragen, denen man erst einmal auf den Grund gehen muss, bevor man sich dem
Design der Software stellen sollte.

Des weiteren gilt es die aktuelle Lage des Marktes zu analysieren. Tools im Bereich der
Geschäftsanalytik gibt es wie Sand am Meer.\cite{WikiBISoftware} Was zeichnen diese Softwarelösungen aus? Mit welchen
Merkmalen kann man sich von vorhandenen Lösungen bewusst unterscheiden? 

Wichtig ist auch die Frage der Art der Frontendapplikation. Was ist eine Progressive Web App und
macht diese im Kontext der Benutzeranforderungen sinn? Was sind Vor- und Nachteile gegenüber
nativer Apps? Wie sieht es mit der unterstützung der Plattformen aus?

Wie könnte die Serverinfrastruktur aussehen? Was für Komponenten sollte Sie beinhalten? Welche Eigenschaften
sollte die Infrastruktur bereitstellen? Mit was für einer Verarbeitungsmenge sollte man rechnen? Wie sieht
es mit der Skalierbarkeit der einzelnen Server aus?

Sind all diese Fragen beantwortet, kann ein Grundkonzept der Gesamtanwendung entwickelt werden.

\section{Wer ist die Zielgruppe?}
Wer ist die Zielgruppe? Die in der Masterarbeit zu entwickelnde Softwarelösungen
soll den Bestandkunden von GBO Datacomp eine moderne Alternative geben, Daten 
auf mobilen Endgeräten wie Tablets, Smartphones aber auch Desktop PCs auswerten zu können.
Desweiteren sollte die Software Neukunden ansprechen und zu dem Kauf überzeugen. GBO Datacomp
hat Kunden in vielen verschiedenen Branchen. So hat das Unternehmen Kunden in der
Automobilindustrie, in der Metall- und Gussindustrie, in der Lebensmittelindustrie,
in der Steinverarbeitung, in der Verpackungsindustrie, in dem Druckgewerbe, in der
Zahntechnik, der Kundstoffverarbeitung, Holzindustrie, Möbelindustrie, Metall- und Gussindustrie sowie
der Steinverarbeitung.\cite{GBODatacompBranchenloesungen}

Eines der wichtigsten Produkte von GBO Datacomp ist BisoftMES, eine Software zur Erfassung von
Maschinen- und Betriebsdaten. Diese Daten können Auftragsdaten, Schichtdaten und Maschinendaten sein.
So können die Benutzer der Software Trendanalysen durchführen, Schicht- und Maschinendaten vergleichen und
Leistungskennzahlen auswerten, um so möglichen Störungen und Ausfällen im Betrieb vorzubeugen.

Wer genau wird also später das agile Dashboard zur Betriebs- und Maschinendatenauswertung
benutzen? Dies kann sich je nach Betrieb unterscheiden. So können beispielsweise von der
Managementebene aus über den Produktionsleiter bis hin zum Schichtarbeiter Benutzer mit dem
Dashboard interagieren. Eines der Dashboards könnte an einem speziell für die Softwarelösung
installierten Bildschirm in einer Produktionshalle angezeigt werden. Anderenfalls könnte aber
auch der Produktionsleiter in der Mittagspause mit seinem Handy die aktuellen KPIs zu deutsch
Leistungskennzahlen abrufen. Viele Produktionshallen besitzen alte Windows-Rechner, welche nicht
mit den neusten Browsern ausgestattet sind. Andere Betriebe besitzen Endgeräte mit Touchscreens.
Fassen wir all diese möglichen Szenarien zusammen, wird uns eines sehr deutlich. Von der
Softwarelösung wird eine gewisse Flexibilität abverlangt, die man nicht unterschätzen sollte.

% Collection Exploration Mugling Modeling Validation Reporting

\section{Anforderungsanalyse}
In der Anforderungsanalyse soll es darum gegen, Anforderungen an die Software zu erkennen,
diese zu erörtern und resultierende Anforderungen auszuwerten. Dabei werden die Anforderungen
in fünf Rubriken unterteilt. Als erstes werden die Anforderungen an die grafische
Benutzeroberfläche gestellt. Hier geht es primär um das Aussehen sowie die Benutzerfreundlichkeit
der Frontendanwendung. Als zweites sollen die Anforderungen an die Funktionalität der
Software analysiert werden. Was sind typische Benutzerhandlungen, die die Software ermöglichen
sollte? Als drittes werden die Anforderungen zur Flexibilität der Datenquellen analysiert. Was
sind mögliche Datensätze, mit denen die Software in Zukunft arbeiten können sollte? Als viertes
werden die möglichen Datenvisualisierungsarten analysiert. Welche Arten von Diagrammen und
Schaubildern benötigen die Benutzer, um aus ihren Daten möglichst viel Information zu ziehen?
Schlussendlich soll das Augenmerk auf die Anforderungen an die Technologien gerichtet werden.

GBO Datacomp hat ihrerseits mögliche Anforderungen an die Software in Form eines Informationsblattes
bereitgestellt. Diese Anforderungen dienen zur Orientierung der System- und Softwaregestaltung. 

\subsection{Gestaltung}
Welche Anforderungen gibt es bei der Gestaltung der Anwendung oder noch spezieller 
der grafischen Oberfläche zu beachten? Wie bereits in der Zielgruppenanalyse
angemerkt, werden die verschiedensten Endgeräte auf die Software zugreifen.
GBO Datacomp erwähnt in ihrem Informationsblatt die fokusierung auf folgende
drei Gerätetypen: Desktop-PCs, Tablets und Smartphones. Man kann also davon ausgehen, dass die
Softwarelösung als Eingabeschnittstelle neben der Maus und der Tastatur auch einen Touchscreen
berücksichtigen sollte. GBO Datacomp weißt speziell auf die unterschiedlichen Bildschirmgrößen
hin. So soll die Anwendung auf verschiedene Bildschirmgrößen reagieren und den Inhalt je nach
Größe zugänglich bereitstellen. Geräte des gleichen Typs sollen keine signifikanten Unterschiede
in der Darstellung aufweisen. Um diese Anforderung umzusetzen, benötigt es neben der Erkennung
der Bildschirmgröße auch noch eine Erkennung des Gerätetyps.

Man kann davon ausgehen, dass neben den unterschiedlichen Gerätetypen auch unterschiedliche
Betriebssysteme verwendet werden. Laut StatCounter verteilt sich der weltweite Marktanteil
an Betriebssystemen laut einer Analyse vom Dezember 2019 wie folgt auf. Android führt mit
40,47\%, gefolgt von Windows mit 34,2\%, iOS mit 14,92\% und macOS mit 7,24\%. Linux und andere
Betriebssysteme bilden hierbei eine Minderheit.\cite{StatCounterOSMarketShare} Betrachtet man den
Nutzungsanteil der Browser im Oktober 2019 weltweit, führt Chrome mit 64,92\%, gefolgt von Safari mit 15,97\%,
Firefox mit 4,33\%, Samsung Internet mit 3,29\%, der primär im asiatischen Bereich genutze Browser UC
mit 2,94\%, Opera mit 2,34\%, Edge mit 2,05\%, Internet Explorer mit 1,98\% und Sonstige.\cite{StatCounterBrowserMarketShare}
Diese Kennzahlen geben einen groben Überblick über die aktuelle Lage des Marktes. Der Markt wird klar von
Google und Microsoft Produkten dominiert. Allerdings variiert der Nutzungsanteil je nach Branche.
Um ein besseres Bild über die verwendeten Technologien möglicher Kunden zu erhalten,
lohnt es sich ein Blick auf die Produkte von GBO Datacomp selbst zu werfen.
BisoftMES, eines der zentralen Produkte von GBO Datacomp, "ist eine unter Microsoft.net
entwickelte Maschinen- und Betriebsdatenerfassungssoftware".\footnote[1]{Aus dem BisoftMES Benutzerhandbuch\cite[S. 7]{BisoftMESHandbuch}}
Es liegt also nahe, dass auch die Bestandskunden von GBO Datacomp Produkte aus der Microsoft
Produktpalette verwenden. Eine Unterstützung von Browsern wie Internet Explorer und Edge
wäre daher wünschenswert.

Die Anwendung sollte als Whitelabel Produkt eingesetzt werden können. Dies bedeutet soviel wie,
dass die GUI der Anwendung mit geringem Aufwand an das Erscheinungsbild eines Unternehmens angepasst
werden kann. Beispiele dafür gibt es in der Softwareindustrie häufig. So kann man Beispielsweise bei
der Community Edition von Gitlab das Logo sowie das Farbschema der Anwendung in den Einstellungen
anpassen.\cite{GitlabDocs}

GBO Datacomp beschreibt in den Anforderungen konkret, dass die Benutzbarkeit der Software
"ohne großen Schulungsaufwand" möglich sein soll. \cite{GBODatacompHandbuchDashboard}
Dies steht im klaren Kontrast zu konkurrierenden Softwarelösungen im BI Sektor.
So bietet QlikTech, ein führendes Unternehmen im Bereich Business Intelligence,
weltweit Kurse für ihr Softwareprodukt Qlik Sense an. Die Kurse haben in der
Regel eine Zeitspanne zwischen drei bis fünf Tagen.\cite{QlikSenseTraining}
Dies ist auch verständlich angesichts der Tatsache, dass QlikTech in einigen
ihrer Produkte beispielsweise eine eigene SQL ähnliche Skriptsprache verwenden.\cite{QlikSenseScriptLanguage}
Um eine klare, selbsterklärende Gestaltung der Anwendung zu erlangen,
muss an der Komplexität der Funktionalität gespart werden. Des weiteren
sollte darauf geachtet werden, das eine klare Abstraktion der Funktionalitäten
gegeben ist. Um den Lernaufwand zu verringern, sollten bekannte Technologien
der Erfindung neuer vorgezogen werden. Falls man jedoch beweisen kann,
dass die neu erfundene Technologie das im Fokus stehende Problem effizienter löst
sowie einfacher zu erlernen ist, spricht nichts dagegen, diese Technologie der bereits
etablierten Technologie vorzuziehen.

Fasst man die oben analysierten Anforderungen in Bezug auf die grafische Benutzeroberfläche
zusammen, kommt man auf folgendes. Die Benutzeroberfläche sollte ein Responsive Design
haben, welches sich auf die Gegebenheiten des jeweiligen Gerätes anpasst. Unterstützung
für die führenden Browser Edge und Internet Explorer inbegriffen sollte gegeben sein.
Das Design sollte mit wenig Aufwand an das Erscheinungsbild eines Unternehmens angepasst werden
können. Die GUI sollte eine klare, benutzerfreundliche Richtlinie verfolgen.

\subsection{Funktionalität}
Die Analyse der Anforderungen an die Funktionalität der Software ist der
umfangreichste Teil der Anforderungsanalyse. Was sind die wichtigsten Anforderungen
an die Funktionalität der Software? Dazu muss man sich den Titel der Masterarbeit
nochmal vor Augen führen. Der Titel heißt: Agiles Dashboardingtool zur Datenvisualisierung
in einer Microservice-Infrastruktur. Die erste und wichtigste Funktionalität sollte also
eine möglichst agile Erstellung eines Dashboards sein. Desweiteren sollten damit Daten
visualisiert werden. Um dies zu ermöglichen, benötigt das Dashboard Daten. Diese Daten
müssen aus einer Datenquelle geladen werden. GBO Datacomp stellt hierfür eine REST API
zur Verfügung. Für eine Datenquelle ist somit gesorgt. Um die Daten innerhalb der Anwendung
einem Dashboard zur Verfügung zu stellen, müssen die Daten zu einem passenden Format verarbeitet werden.
Zu guter Letzt muss noch definiert werden, welche Daten in welchem Dashboard verwendet
werden sollten. Diese vier Funktionalitäten sind der Kern der Anwendung. In diesem Abschnitt
werden zuerst die Anforderungen der Kernfunktionalitäten genauer analysiert. Daraufhin werden die Anforderungen
für die Authentifizierung der Benutzer, die Rechteverwaltung, die Auswahl der Sprache,
die Navigation durch die Software, die Suche nach Ressourcen in der Anwendung, einfaches
Erstellen von Informationsseiten sowie für die Möglichkeit, die Anwendung auch offline
zu verwenden, erörtert.

Was sind die Anforderungen für die Erstellung des agilen Dashboards? Wie bereits in dem
vorigen Abschnitt über die grafische Benutzeroberfläche debatiert, sollte die Anwendung und
so auch die Erstellung des Dashboards selbsterklärend sein. Daraus resultiert, dass die Anwendung 
selbst das genaue Zentrieren der einzelnen Komponenten des Dashboards übernehmen sollte. Der
Benutzer sollte in einem intuitiven Verfahren, einzelne Komponenten auswählen und diese
an bestimmte Positionen im Bildschirm ziehen können. Desweiteren sollte er Komponenten einfach
entfernen und ersetzen können. Wirft man einen Blick auf andere BI Softwarelösungen,
kann man feststellen, dass die Anzahl der Komponenten der Anwendung im Lauf der Zeit
stetig anwachsen kann. Es soll also auch möglich sein, von einer großen Anzahl an Komponenten
bequem eine passende Komponente zu finden und diese dann zu verwenden.

Was sind die Anforderungen für das Laden der Daten in die Anwendung? Neben der Flexibilität,
die im darauffolgenden Abschitt erörtert wird, sollte das Laden von Daten schnell und zuverlässig
sein. Typische Vorgehensweisen der Anwender sind das Auswerten von Maschinendaten. Dabei kann es
sich um die Auslastung, Störfälle aber auch Ausfälle der Maschinen handeln. Es ist also sehr wichtig,
das der Anwender rechtzeitig über Stör- und Ausfälle der Maschinen informiert wird. Unsere
Aufmerksamkeit gilt also auch der zeitnahen Aktualisierung der Daten in den Dashboards. Nur
so kann sichergestellt werden, dass der Anwender auch zeitnah auf in den Daten erkenntliche
Ereignisse reagieren kann. In Anlehnung dieser Feststellung liegt es nahe, den Benutzer
über bestimmte Ereignisse per Benachrichtigung zu informieren. So könnte man beispielsweise
eine Benachtichtigung auf das Smartphone des Benutzers schicken, sobald ein zuvor definierter
Schwellwert überschritten wird.

Welche Anforderungen sind für die Verarbeitung der Daten notwendig? Daten können Fehler oder Lücken
beinhalten. So kann in einer Spalte, die das ISO Datumsformat beinhalten, fälschlicherweise ein
ungültiger Wert vorkommen oder dieser komplett fehlen. Zugegebenermaßen reicht es für einen MVP, ein
minimal überlebensfähiges Produkt, aus, diese Unreinheiten so zu behandeln, dass die Anwendung
eine informationsreiche Fehlerwarnung an den Benutzer überliefert oder alternativ einen Standartwert
für fehlende Felder definitert.

Wie werden die Dashboards mit den nötigen Daten versorgt? Grundsätzlich ist klar davon auszugehen,
dass die Anwendung einen Zuweisungsmechanismus bereitstellt. Um von einer komplexeren, eingehenden
Datenstruktur zu einer für die Visualisierung geeigneten Datenstruktur zu kommen, müssen die gewollten
Daten ausgewählt und zugewiesen werden. Dem Benutzer selbst sollte ein, ähnlich wie bei der Erstellung
des Dashboards, agiler Prozess zur Zuweisung der Daten ermöglicht werden. Ein Benutzer will auf einem
Dashboard unterschiedliche Quellen von Daten vergleichen können. Somit muss die Möglichkeit bestehen,
unterschiedliche Zuweisungen für die Diagramme eines Dashboards zu tätigen.

Anschließend werden die Anforderungen für die Benutzerauthentifizierung und die Rechteverwaltung erörtert.
Sicherheit sollte in jeder Anwendung, welche mit Kundendaten interagiert, von großer Bedeutung sein.
Somit ist es naheliegend, dass für die heutigen IT-Sicherheitsstandards gesorgt werden sollte. Progressive
Web Apps sind beispielsweise per Definition außerhalb der Entwicklung nur über HTTPS zu übertragen.\cite[S. 16]{KevinFrankPWAMasterarbeit}
OWASP (Open Web Application Security Project) veröffentlichte 2017 einen Bericht über die zehn
wichtisten Sicherheitsrisiken von Webanwendungen. Darunter sind unter anderem die Einschleusung
von Quellcode durch mangelnde Maskierung, fehlerhafte Authentifizierungsverfahren, Preisgabe von sensiblen Daten,
XSS-Attacken und Fehlkonfigurationen von Sicherheitseinstellungen.\cite[S. 4]{OWASPTopTen}
Um die Kundendaten vor Angriffen zu schützen, liegt es nahe, diese Sicherheitsrisiken bei der
Entwicklung der Anwendung beachtet werden müssen. 

\subsection{Flexibilität}
% Anforderungen an die Flexibilität der Datenquellen
% Unterschiedliche Datenquellen
% der Datenquellen, der Datenart

% Visualisierung der Daten -> WICHTIG: VISUALISIERUNG DER DATEN
% Anforderungen an die Visualisierung der Daten
% Mögliche Verabschaulichung der Daten
% Aktuelle Modelle: Measures and Dimensions

% KPI

\subsection{Codebasis}
% Anforderungen an die Codebasis
% Einstieg für andere Entwickler
% Weiterverwendbarkeit und Erweiterbarkeit
% Lizenz
% Continuous Integration and Continuous Delivery
% Tests

\section{Auswertung vorhandener Softwarelösungen}
2018 verteilte sich mehr als 50\% des weltweiten Marktanteils an BI- und Geschäftsanalytiksoftware auf
zehn Anbieter. Den größten Marktanteil besitzt Microsoft mit 10,6\%, gefolgt von SAP, IBM, SAS, Oracle,
Tableau Software und Qlik.\cite{StatistaMarketshareBI}
In dieser Arbeit werden drei Softwarelösungen analysiert und ausgewertet, darunter Power BI Desktop von
Microsoft, Qlik Sense von Qlik und Tableau von Tableau Software. Die Auswahl der drei Softwarelösungen
basiert auf dem Marktanteil des Anbieters und der zu Verfügung stehenden Literatur. Alle diese 
Anwendungen ermöglichen die Erkundung, Bereinigung und Auswertung von Daten. Was aber unterscheidet
die Anwendungen von einander? Wo liegen ihre Stärken und Schwächen und welche von den zuvor genannten
Anforderungen werden von den Marktführern erfüllt?

\subsection{Power BI Desktop}

% Wie hat es Power BI gelöst

\subsection{Qlik Sense}
% Wie hat es Qlik Sense gelöst

\subsection{Tableau}
% Wie hat es Tableau gelöst

\section{Progressive Web App}
% PWA vs Native
Eine Progressive Web App ist eine Webseite, die eine zunehmende Ansammlung an bestimmten Technologien,
Strategien und Schnittstellen implementiert, welche Merkmale bereitstellen, die ursprünglich nur
native Apps besaßen.\cite{WikiPWA} Eine der wichtigsten Browser-Technologien, die zur Entstehung der PWA beigetragen hat,
ist der Service Worker. Er ermöglicht der PWA, auch ohne Internetverbindung zu funktionieren. Der 
Service Worker ist ein auf einem separaten Thread laufender Prozess, welcher HTTP und je nach Browser
auch WebSocket Anfragen abfangen und verarbeiten kann. Somit kann trotz fehlernder Internetverbindung
eine Antwort zurückgesendet werden. \cite{W3ServiceWorker}

Progressive Web Apps als Alternative zur nativen App ermöglichen mehr Freiheit gegenüber dem
Plattformanbieter. Sobald die Möglichkeit besteht, PWAs mit Hilfe einer vom Browser angesteuerten API
auf dem Homescreen zu speichern, verliert der Plattformanbieter die Kontrolle,
bestimmte PWAs zu verbieten. Dies ist bei nativen Apps nicht der fall. So löschte Apple in der Vergangenheit
bereits große Mengen an Apps aus dem App Store.\footnote[1]{Laut Heise.de löschte Apple im Oktober 2016 rund 47.300 Apps aus dem App Store.\cite{HeiseAppleLoeschtApps}} 
Des weiteren verlieren die Plattformanbieter die Gewinnbeteiligung an App-, Abo- und Ingame-Verkäufen,
da die Zahlungsabwicklung nun nicht mehr über den vom Plattformanbieter bereitgestellten App Store
abgewickelt werden muss. Entwickler von Apps mussten in der Vergangenheit bis zu 30\% 
ihrer Einnahmen abgeben. \cite{WinFutureEigenerAppStore} Nichtsdestotrotz arbeiten
die Plattformanbieter weiter an neuen Features für PWAs.

% Vorteil für kleines Entwicklerteam, da eine Lösung für alles

\subsection{Merkmale einer PWA}
% Bestandteilanalyse

\subsection{Unterstützter Funktionsumfang gängiger Browser}
% PWA Feature Detection

\section{Microservices}

\subsection{Was versteht man unter Microservices?}
% Analyse Microservices

\subsection{Sind Microservices eine zukunftsfähige Softwarearchitektur?}
% Microservices vs SOA

\section{Zielsetzung}
% Kurze zusammenfassung über das Ziel des Projekts

\section{Entwurf}

\subsection{Skizzen der Benutzeroberfläche}
% Wireframes
% Agile zusammenstellen eines Dashboards
% Agiles Datenzuordnung in einem Datenflussdiagramm
% die agile Erstellung eines Datenflussdiagramms, welches die Daten aus den Datenquellen verarbeitet und den Dashboards zuweist.
\subsection{Skizzen der Infrastruktur}

% Wahl der Technologien, Tech-Stack?
% Anforderungen an die Technologien
% Umfeld etc
% Geschwindigkeit der Entwicklung
