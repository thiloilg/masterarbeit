\chapter{Konzeption}

Zur Erstellung des Konzepts gilt es folgende Aspekte im voraus genauer zu betrachten.
Wer ist meine Zielgruppe? Was ist für die zukünftigen Benutzer der Software wichtig?
Was ist die Anforderungen an die Software? Gibt es bestimmte Abläufe, die von der
Software bewerkstelligt werden müssen? Wie flexibel muss die Software auf Änderungswünsche
reagieren können? Was ist bei der Gestaltung der Benutzeroberfläche wichtig? 
Welcher Tech-Stack wird den Anforderungen gerecht? Was für Daten werden die Kunden mit der
Software auswerten? Wie und in welcher Form sollten die Daten dargestellt werden? 
Fragen über fragen, denen man erst einmal auf den Grund gehen muss, bevor man sich dem
Design der Software stellen sollte.

Des weiteren gilt es die aktuelle Lage des Marktes zu analysieren. Tools im Bereich der
Geschäftsanalytik gibt es wie Sand am Meer. Was zeichnen diese Softwarelösungen aus? Mit welchen
Merkmalen kann man sich von vorhandenen Lösungen bewusst unterscheiden? 

Wichtig ist auch die Frage der Art der Frontendapplikation. Was ist eine Progressive Web App und
macht diesen im Kontext der Benutzeranforderungen sinn? Was sind Vor- und Nachteile gegenüber
nativer Apps? Wie sieht es mit der unterstützung der Plattformen aus?

Wie könnte die Serverinfrastruktur aussehen? Was für Komponenten sollte Sie beinhalten? Welche Eigenschaften
sollte die Infrastruktur bereitstellen? Mit was für einer Verarbeitungsmenge sollte man rechnen? Wie sieht
es mit der Skalierbarkeit der einzelnen Server aus?

Sind all diese Fragen beantwortet, kann ein Grundkonzept der Gesamtanwendung entwickelt werden.

\section{Wer ist die Zielgruppe?}
Wer ist die Zielgruppe? Die in der Masterarbeit zu entwickelnde Softwarelösungen
soll den Bestandkunden von GBO Datacomp eine moderne Alternative geben, Daten 
auf mobilen Endgeräten wie Tablets, Smartphones aber auch Desktop PCs auswerten zu können.
Desweiteren sollte die Software Neukunden ansprechen und zu dem Kauf überzeugen. GBO Datacomp
hat Kunden in vielen verschiedenen Branchen. So hat das Unternehmen Kunden in der
Automobilindustrie, in der Metall- und Gussindustrie, in der Lebensmittelindustrie,
in der Steinverarbeitung, in der Verpackungsindustrie, in dem Druckgewerbe, in der
Zahntechnik, der Kundstoffverarbeitung, Holzindustrie, Möbelindustrie, Metall- und Gussindustrie sowie
der Steinverarbeitung.

Eines der wichtigsten Produkte von GBO Datacomp ist BisoftMES, eine Software zur Erfassung von
Maschinen- und Betriebsdaten. Diese Daten können Auftragsdaten, Schichtdaten und Maschinendaten sein.
So können die Benutzer der Software Trendanalysen durchführen, Schicht- und Maschinendaten vergleichen und
Leistungskennzahlen auswerten, um so möglichen Störungen und Ausfällen im Betrieb vorzubeugen.

Wer genau wird also später das agile Dashboard benutzten, um die Betriebs- und Maschinendaten
auszuwerten? Dies kann sich je nach Betrieb unterscheiden. So können beispielsweise von der
Managementebene aus über den Produktionsleiter bis hin zum Schichtarbeiter Benutzer mit dem
Dashboard interagieren. Eines der Dashboards könnte an einem speziell für die Softwarelösung
installierten Bildschirm in einer Produktionshalle angezeigt werden. Anderenfalls könnte aber
auch der Produktionsleiter in der Mittagspause mit seinem Handy die aktuellen KPIs zu deutsch
Leistungskennzahlen abrufen wollen. Viele Produktionshallen alte Windows-Rechner, welche nicht
mit den neusten Browsern ausgestattet sind. Andere Betriebe besitzen Endgeräte mit Touchscreens.
Fassen wir all diese möglichen Szenarien zusammen, wird uns eines sehr deutlich. Von der
Softwarelösung wird eine gewisse Flexibilität abverlangt, die man nicht unterschätzen sollte.

\subsection{Grafische Benutzeroberfläche}
Welche Anforderungen gibt es bei der grafischen Oberfläche zu beachten? Wie bereits
in der Zielgruppenanalyse angemerkt, werden die verschiedensten Endgeräte auf die
Software zugreifen. Darunter zählen unter anderem Smartphones sowohl Android als auch IOS,
Tablets und Desktop PCs mit entweder Windows, MacOS als auch Linux. Die grafischen Elemente
der Software sollen sich an die Größe des Bildschirms anpassen und den Platz des Bildschirms
bestmöglich nutzen. Die Andendung sollte in horizontaler als auch verticaler Ausrichtung des
Bildschirms verwendet werden können. In der Softwareentwicklung spricht man hier von dem
sogenannten Responsive Design, welches durch zahlreiche Webframeworks und Bibliotheken
unterstützt wird.

Die Anwendung sollte als Whitelabel Produkt eingesetzt werden können. Dies bedeutet soviel wie,
dass die GUI der Anwendung mit geringem Aufwand an das Erscheinungsbild eines Unternehmens angepasst
werden kann. Beispiele dafür gibt es in der Softwareindustrie häufig. So kann man Beispielsweise bei
der Community Edition von Gitlab das Logo sowie das Farbschema der Anwendung in den Einstellungen
anpassen. \cite{GitlabDocs} % cication for https://docs.gitlab.com/ee/user/admin_area/appearance.html

\subsection{Technologien}
% Anforderungen an die Technologien
% Umfeld etc
% Geschwindigkeit der Entwicklung

\subsection{Funktionalitäten der Software}
% Anforderungen an die Funktionalität der Software

\subsection{Flexibilität der Datenquellen}
% Anforderungen an die Flexibilität der Datenquellen

\subsection{Visualisierung der Daten}
% Anforderungen an die Visualisierung der Daten

\section{Auswertung vorhandener Softwarelösungen}

\subsection{Power BI}
% Wie hat es Power BI gelöst

\subsection{Qlik Sense}
% Wie hat es Qlik Sense gelöst

\subsection{Tableau}
% Wie hat es Tableau gelöst

\section{Ist eine Progressive Web App geeignet?}

\subsection{Was ist eine Progressive Web App?}

\subsection{Unterstützter Funktionsumfang der gängigen Browser}

\section{Entwurf}

\subsection{Skizzen der Benutzeroberfläche}
% Wireframes

\subsection{Skizzen der Infrastruktur}
