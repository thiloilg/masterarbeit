\chapter{Ausblick}
\label{chap:ausblick}

Die Gesamtanwendung bietet ein solides Grundgerüst. Jede der einzelnen Komponenten
kann an Funktionalitäten angereichert werden. Im Hinblick auf die nahe Zukunft, bietet es sich an,
weitere Bündel an Plugins zur Datenvisualisierung zur Verfügung zu stellen. Hier könnten
zu den bereits existierenden, rudimentären Diagrammen auch weitere Visualisierungsarten ermöglicht
werden.

Für die benutzerfreundliche Erstellung einer Datenquelle und derer Verarbeitung
sollte eine Validierung eingebaut werden, die dem Benutzer bescheid gibt, ob das Datenformat
gültig ist. Die Eingangsdaten der Plugins können sehr flexibel sein. Hier muss an einem Standart
gearbeitet werden, der den Aufbau der Daten klar definiert, allerdings auch genug Freiheit bietet,
um komplexe Datenstrukturen darzustellen.

Aus Sicht der Performanz des Systems ist der nächste
Schritt, statischen Dateien nicht mehr über eigene Nginx-Webserver sondern ein
Content Delivery Netzwerk auszuliefern. Dies dient dazu, das eigene Servercluster
zu entlasten und die globalen Ladezeiten der Frontendanwendung zu verringern.
Um die PWA im Apple App Store zu veröffentlichen, gilt es, eine React Native
App mit einem eingebetteten WebView zu implementieren. Anders als bei den TWAs, muss
diese stetig aktualisiert werden. Dies ist nötig, um die Browserversion auf
dem neuesten Stand zu halten.

Der grundsätzliche Plan für die Zukunft ist es, sich stärker an den
Bedürfnissen der Kunden zu orientieren, um Funktionalitäten der Anwendung
so benutzerfreundlich wie möglich zu gestalten.
