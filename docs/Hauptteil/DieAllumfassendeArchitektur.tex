\chapter{Die allumfassende Architektur}
\label{chap:die-allumfassende-architektur}

\section{Die Aufteilung der Logik}
\label{sec:die-aufteilung-der-logic}

\section{Das Design des Datenflusses}
\label{sec:das-design-des-datenflusses}

\section{Die Definition der Schnittstellen}
\label{sec:die-definition-der-schnittstellen}

\section{Die Frage der Dokumentation}

\section{Die Wahl der Technologien}

\section{Die Bewahrung der Offenheit gegenüber der Veränderung}
\label{sec:die-bewahrung-der-offenheit-gegenueber-der-veraenderung}

\chapter{Der Durchstich}

\section{DevOps ist dein bester Freund}

\section{Die Unabhängigkeit gegenüber einzelner Unternehmen}
\subsection{Der eigene Gitlab Server}

\section{Wie richte ich meine Pipeline ein?}

\section{Der richtige Umgang mit Umgebungsvariablen}

\section{Die Gesundheit der Services}
In Microservice-Infrastrukturen sind die einzelnen Services eng miteinander verküpft
und voneinander abhängig. Um die Verfügbarkeit sicherzustellen, werden auf
Serviceebene Healthcheckendpunkte bereitgestellt. Diese spiegeln den aktuellen
Gesundheitszustand des jeweiligen Dienstes wieder. Ein Healthcheckendpunkt
ist speziell dann nützlich, wenn ein Dienst oder ein Verfahren von einem anderen
abhängig ist. So ist beispielsweise die Backendapi von der Datenbank
abhängig. Andererseits ist der Frontenddiesnt sowie das Apitestverfahren
in der Buildpipeline von der Backendapi abhängig. 

\begin{listing}
    \label{lst:healthcheck}
    \inputminted{sh}{snippets/sh/healthcheck.sh}
    \caption{Healthcheckbeispiel in Docker}
\end{listing}

\section{Die Unabhängigkeit des Betriebssystems}
\subsection{Multistage Builds}

\section{Wer automatisiert, gewinnt auf lange Zeit}

\chapter{Die integrierte Qualitätssicherung}

\section{Die testgetriebene Entwicklung}

Finster war's, der Mond schien helle auf die grünbeschneite Flur, als
ein Wagen blitzesschnelle langsam um die runde Ecke fuhr. Drinnen
saßen stehend Leute schweigend ins Gespräch vertieft, als ein
totgeschossner Hase auf dem Wasser Schlittschuh lief und ein
blondgelockter Knabe mit kohlrabenschwarzem Haar auf die grüne Bank
sich setzte, die gelb angestrichen war.
Finster war's, der Mond schien helle auf die grünbeschneite Flur, als
ein Wagen blitzesschnelle langsam um die runde Ecke fuhr. Drinnen
saßen stehend Leute schweigend ins Gespräch vertieft, als ein
totgeschossner Hase auf dem Wasser Schlittschuh lief und ein
blondgelockter Knabe mit kohlrabenschwarzem Haar auf die grüne Bank
sich setzte, die gelb angestrichen war.

\section{CI, CD und nochmal CD}
\label{sec:ci-cd-und-nochmal-cd}


Finster war's, der Mond schien helle auf die grünbeschneite Flur, als
ein Wagen blitzesschnelle langsam um die runde Ecke fuhr. Drinnen
saßen stehend Leute schweigend ins Gespräch vertieft, als ein
totgeschossner Hase auf dem Wasser Schlittschuh lief und ein
blondgelockter Knabe mit kohlrabenschwarzem Haar auf die grüne Bank
sich setzte, die gelb angestrichen war.
Finster war's, der Mond schien helle auf die grünbeschneite Flur, als
ein Wagen blitzesschnelle langsam um die runde Ecke fuhr. Drinnen
saßen stehend Leute schweigend ins Gespräch vertieft, als ein
totgeschossner Hase auf dem Wasser Schlittschuh lief und ein
blondgelockter Knabe mit kohlrabenschwarzem Haar auf die grüne Bank
sich setzte, die gelb angestrichen war.


\subsection{Continuous Delivery}
\label{subsec:continuous-delivery}
Wenn Sie bei der Abkürzung CD immernoch an eine Compact Disc denken, muss ich Sie leider enttäuschen. Jene ist ein veralteter optischer Speicher, welcher heutzutage nur noch in Computermuseen oder Rückdeckeln diverser wissenschaftlicher Arbeiten zu finden ist. Aber darum geht es hier garnicht. Was ich damit meine, ist Continuous Delivery, zu deutsch stetige Softwareauslieferung.


\section{Die Zentrierung service-relevanter Einstellungen}
% extrahierung von theme colors etc.

\section{Was für Tests gibt es, und wenn ja, wieviele?}

\section{SCRUM im Alleingang}

\section{Zwing dich zu deinem Glück}

\subsection{Die Einrichtung von Githooks}
\section{Testing Stages}

\section{Die Projektstruktur}

\section{Die Arbeitsablaufoptimierung}

\section{Die standige Reduzierung von Abhängigkeiten}

\section{Das allseits beliebte Refactoring}

\section{Wann ist Performance wichtig?}

\section{MVP}

\chapter{Die Sicherheit}
\section{JWT}
\section{SSO}
\section{Two-factor Auth}
\section{Two-Way Validation}
\section{Die Rechteverwaltung}
\section{Whitelist statt Blacklist}
\section{One verification route}

\chapter{Die Implementierungsdetails}

\section{Singleton, das Anti-Pattern}
\section{Was sind PWAs?}

\subsection{Risiken für Unternehmen}

\subsection{Die Bestandteilanalyse}

\section{Der Service Worker}
Ein Service Worker ist ein in JavaScript geschriebener,
eventbasierter Proxy, welcher in einem separaten Thread im Browser
läuft und an eine spezifische Origin gekoppelt ist. Er hat
die Möglichkeit Anfragen zwischen der im Browser ausgeführten
Anwendung und dem Server abzufangen und zu verarbeiten.

\subsection{Die wahl der Zwischenspeicherstrategie}

\section{Batch Requests}

\section{Filter specific fields}

\chapter{Die Feature Requests}

für eine solide Softwarearchitektur

Finster war's, der Mond schien helle auf die grünbeschneite Flur, als
ein Wagen blitzesschnelle langsam um die runde Ecke fuhr. Drinnen
saßen stehend Leute schweigend ins Gespräch vertieft, als ein
totgeschossner Hase auf dem Wasser Schlittschuh lief und ein
blondgelockter Knabe mit kohlrabenschwarzem Haar auf die grüne Bank
sich setzte, die gelb angestrichen war.

\section{Was versteht man unter CD?}
\label{sec:was-versteht-man-unter-cd}


\begin{listing}
    \label{lst:HelloJSX}
    \caption{Ein einfaches JSX Beispiel}
    \inputminted{jsx}{snippets/examples/Welcome.jsx}
\end{listing}

\begin{listing}
    \label{lst:Golang}
    \caption{Ein einfaches Golang Beispiel}
    \inputminted{go}{snippets/examples/hello.go}
\end{listing}

\newpage

\begin{figure}
    \label{figure:test}
    \includegraphics[scale=0.2]{img/HTW}
    \caption{Test}
\end{figure}

\begin{figure}
    \label{figure:beispiel}
    \includegraphics[scale=0.2]{img/HTW}
    \caption{Beispiel}
\end{figure}
