\chapter{Die allumfassende Architektur}
\label{chap:die-allumfassende-architektur}

\section{Auswertung vorhandener Produkte}
% Qlik, Tableau, Grafana => Vergleichem

\section{Workflow of a Data Scientist}
% Collection Exploration Mugling Modeling Validation Reporting

\section{Mögliche Veranschaulichungen von Daten}
% Aktuelle Modelle: Measures and Dimensions

\section{Mircoservices}

\subsection{Microfrontend}
Domain Driven Development also in Frontend.

Aufteilung in Module, die während der laufzeit geladen werden.
Seite 171 in Buch (Eberhard Wolff Microservices) befasst sich mit dem
Thema, wenn auch nicht auf das dynamische Laden eingegangen wird.
"Das Deployment der SPA ist meistens nur als vollständige Anwendung möglich."
Neuer Ansatz für unabhängiges Deployment!

-> Auch Problem: Deployment Monolith -> Separation der Deployment Vorgänge

-> "Wenn in einem Microservice ein Feature umgesetzt wird, das auch Änderungen
in der Client-Anwendung benötigt, kann diese Änderung nicht durch eine neue
Version des Microservice alleine ausgerollt werden. Es muss auch eine neue Version
der Client-Anwendung ausgeliefert werden." Seite 176.  (Eberhard Wolff Microservices)
Absatz 3.

Deployment abhängigkeiten verringern!

SOAP Schnittstelle zwischen Frontend Chart Plugins und Frontend Arrangement Software!
Weniger bis garkeine Abhängigkeit von Implementierungsdetails

Problem: Steigende Komplexität der Anwendung.

\section{Die Schnittstellengestaltung}
% sehr wichtig, abstraktion

\section{Die Aufteilung der Logik}
\label{sec:die-aufteilung-der-logic}
Separation of Concerns, etwa Trennung der Zuständigkeiten

\section{Die Datenstruktur}
Da die Daten und deren Beziehungen von Anwendungsfall zu Anwendungsfall
variieren, gilt es eine klar definierte Datenstruktur zu finden, die
die gängigsten Anwendungsfälle abdecken kann. 

% Show dashboard json structure using maybe css grid
% and data structure to pass to plugins
% many to many to many relations???

\section{Das Design des Datenflusses}
\label{sec:das-design-des-datenflusses}

\section{Die Definition der Schnittstellen}
\label{sec:die-definition-der-schnittstellen}

\section{Die Wahl der Technologien}

\subsection{React}
Dan Abramov Representational components vs container components => actual view and controller


\section{Die Bewahrung der Offenheit gegenüber der Veränderung}
\label{sec:die-bewahrung-der-offenheit-gegenueber-der-veraenderung}

\subsection{Die Änderung des Quellcodes}

\subsection{Das stetige Aktualisieren der ausgelieferten Software}
Probleme: cookies verändern sich, serviceworker verändert sich, neue Felder in der Datenbank
Frontend und Backend etc, also allen verschiedenen Services


\begin{listing}
    \label{lst:HelloJSX}
    \caption{Ein einfaches JSX Beispiel}
    \inputminted{jsx}{snippets/examples/Welcome.jsx}
\end{listing}

\begin{listing}
    \label{lst:Golang}
    \caption{Ein einfaches Golang Beispiel}
    \inputminted{go}{snippets/examples/hello.go}
\end{listing}

\newpage

\begin{figure}
    \label{figure:test}
    \includegraphics[scale=0.2]{img/HTW}
    \caption{Test}
\end{figure}

\begin{figure}
    \label{figure:beispiel}
    \includegraphics[scale=0.2]{img/HTW}
    \caption{Beispiel}
\end{figure}
