\chapter{Grundlagen}
\label{chap:grundlagen}
In den Grundlagen werden die für die Arbeit relevanten Themen, progressive Webanwendungen in \Cref{sec:progressivewebanwendung},
Microservices in \Cref{sec:microservices} und kontinuierliche Integration und Auslieferung
in \Cref{sec:kontinuierlicheintegrationundauslieferung} behandelt. Da die Arbeit teilweise noch
sehr junge Technologien verwendet, beinhaltet das Grundlagenkapitel in den einzelnen Abschnitten
auch Diskussionen, Analysen und Erörterungen der einzelnen Thematiken.

\section{Progressive Webanwendung}
\label{sec:progressivewebanwendung}
Eine progressive Webanwendung (PWA) ist eine Webseite, die eine zunehmende Ansammlung
an bestimmten Technologien, Strategien und Schnittstellen implementiert,
die Merkmale bereitstellen, die ursprünglich nur native Apps besaßen. \cite{WikiPWA}
Eine der wichtigsten Browser-Technologien, die zur Entstehung der PWA beigetragen hat,
ist der Service Worker (SW). Er ermöglicht der PWA auch ohne Internetverbindung zu funktionieren. Der 
Service Worker ist ein parallel zu der Hauptanwendung laufender Prozess, der \mbox{HTTP-,} und je nach Browser,
auch WebSocket-Anfragen, abfängt, verarbeiten und versenden kann. Somit ist es möglich, trotz fehlender Internetverbindung
eine Antwort an das Frontend zurückzusenden.\cite{W3ServiceWorker}

Eine progressive Webanwendungen als Alternative zu einer nativen Umsetzung
ermöglichen größere Freiheit gegenüber den Plattformanbietern. Sobald die
Möglichkeit besteht, PWAs mithilfe einer vom Browser angesteuerten Programmierschnittstelle (API)
auf dem Startbildschirm zu speichern, verliert der Plattformanbieter die Kontrolle,
bestimmte PWAs zu verbieten. Dies ist bei nativen Anwendungen nicht der Fall. So löschte Apple in der Vergangenheit
bereits große Mengen an Apps aus dem App Store.\footnote{Laut Heise.de löschte Apple im Oktober
2016 rund 47.300 Apps aus dem App Store \cite{HeiseAppleLoeschtApps}.}
Des Weiteren verlieren die Plattformanbieter die Gewinnbeteiligung an App-, Abo- und Ingame-Verkäufen,
da die Zahlungsabwicklung nicht mehr über die von den Plattformanbietern bereitgestellten App Stores
abgewickelt werden muss. Entwickler von Apps mussten in der Vergangenheit bis zu 30\% 
ihrer Einnahmen abgeben.\cite{WinFutureEigenerAppStore} Daher ist davon auszugehen,
dass Plattformanbieter ihren Fokus für neue Funktionalitäten primär ihren eigenen
nativen Apps widmen werden.

\mbox{Nichtsdestotrotz} hat sich in letzter Zeit einiges getan.
Gerade die Bandbreite der Plattform- und Browseranbieter, die für PWAs relevante
Funktionalitäten anbieten, ist in der Vergangenheit stark gestiegen. So äußert sich der Artikel
"Webdesign: So steht es um Progressive-Web-Apps im Jahr 2020" von Dieter Petereit,
der am 20. Januar 2020 auf t3n.de veröffentlicht wurde, über die Unterstützung der PWAs wie folgt:

\begin{quote}
"Immerhin 93 Prozent aller Nutzer im weltweiten Netz verwenden einen Browser, der vollen Support
für PWA bietet. Installieren könnten PWA immerhin bereits rund 86 Prozent aller Nutzer. Auch
die Zielplattform der Installation ist nahezu frei wählbar. Insbesondere funktioniert es
unter Windows ab Version 7, macOS, Linux, Xbox One, Android, iOS, iPadOS und natürlich Chrome OS,
allerdings nicht mit jedem Browser."\cite{T3NPWASupport}
\end{quote}

Aus technologischer Sicht sind die
Anforderungen, die es benötigt, um eine PWA zu benutzen, größtenteils gegeben.
Aufgrund dessen davon auszugehen, dass PWAs native Apps ersetzen werden, ist allerdings
falsch. Erst wenn Plattformanbieter einen Weg finden, finanziell von PWAs profitieren
zu können, werden die Funktionalitäten in einem Maße unterstützt, die ihnen die Möglichkeit
gibt, mit nativen Apps mithalten zu können. Der Schritt von einer Single-Page-Webanwendung (SPA) zu einer
PWA ist allerdings ein leichter.\footnote{Wenn man sich auf die Hauptfunktionalitäten beschränkt.}
Um die PWA auf dem Startbildschirm speichern zu können, muss man ein Web-App-Manifest\footnote{
Ein Beispiel eines Web-App-Manifests befindet sich unter Weitere Quellcodeausschnitte \ref{lst:webappmanifest}.}
in das Stammverzeichnis des Webservers legen sowie einen Service Worker registrieren.
Somit ist die Einstiegsebene für PWAs sehr gering. Gerade für Unternehmen,
die bereits eine Webseite besitzen und dem Benutzer die Möglichkeit geben wollen,
diese auf der Benutzeroberfläche zu speichern, offline zu verwenden und Benachrichtigung zu erhalten,
ist eine PWA attraktiv.

Ein Hauptkriterium für die Implementierung einer PWA ist, dass man bei der Entwicklung
keine plattformspezifischen Lösungen benötigt. Somit sind die Kosten für die Entwicklung um einiges
geringer. Verfügt man allerdings über die finanziellen Mittel und legt seinen Schwerpunkt unter anderem
auf mobile Endgeräte, kann es durchaus Sinn machen, neben einer aufgewerteten Webseite
auch native Apps anzubieten.

In diesem Abschnitt wurden Merkmale, wie das Speichern der Webseite auf der Benutzeroberfläche,
die Offlinefähigkeit sowie die Möglichkeit, Benachrichtigungen zu versenden, kurz angesprochen.
In den folgenden drei Abschnitten wird vermehrt ins Detail gegangen: In \Cref{subsec:markmale}
werden die Merkmale einer PWA erörtert. Der unterstützte Funktionsumfang der zuvor erläuterten Merkmale 
wird in \Cref{subsec:unterstuetzterfunktionsumfanggaengigerbrowser} anhand unterschiedlicher Browser 
analysiert. In \Cref{subsec:trustedwebactivities} geht es um eine von Google vorgestellte Methode,
die es ermöglicht PWAs in den App Store zu bringen.

\subsection{Merkmale}
\label{subsec:markmale}
Es gibt keine genaue Definition, die besagt, welche Merkmale eine PWA beinhalten muss.
Der Begriff "Progressive Web App" wurde erstmals 2015 von Alex Russell in einem Blogpost
verwendet.\cite{PWA2015} Die Idee hinter einer Webanwendung bei der sich die Optik und Haptik,
wie die bei einer nativen App anfühlen, gab es allerdings bereits zuvor. So kündigte
Steve Jobs bereits 2007 bei der initialen iPhone Präsentation für Entwickler bereits die Möglichkeit
an, solche Web 2.0 Apps mithilfe von Safari zu entwickeln.\cite{SteveJobsIphonePresentation}
Die Merkmale einer PWA haben sich demnach über die Jahre hinweg angesammelt. Was sind aber
nun diese Merkmale? Alex Russell erwähnt in seinem Blogpost dazu Folgendes:
Die PWA soll responsive und offline verwendbar sein, app-ähnliche Navigationsvorgänge bereitstellen,\footnote{Ohne Verwendung von Browser Tabs.} 
immer aktuell sein, nur unter Verwendung der Transport Layer Security (TLS) ausgeliefert werden, durch Suchmaschinen als solche mithilfe
des Web-App-Manifests auffindbar sein, plattformspezifische Funktionalitäten wie Push-Notifications
verwenden sowie auf dem Startbildschirm installierbar, über Hyperlinks teilbar und ohne aufwendigen
Installationsprozess verwendbar sein.\cite{PWA2015} Eine 2018 veröffentlichte wissenschaftliche
Arbeit von Thomas Steiner, über die Analyse von Merkmalen einer PWA innerhalb eines WebViews\footnote{Ein Web View ist ein in einer nativen App eingebetteter Browser zur Darstellung von Webinhalten.},
benennt fünfzehn für eine PWA relevante Merkmale. Die Analyse über den unterstützten Funktionsumfang
von gängigen Browsern in \Cref{subsec:unterstuetzterfunktionsumfanggaengigerbrowser}
bezieht sich auf diese Merkmale.\cite{WhatIsInAWebView}

\subsection{Unterstützter Funktionsumfang gängiger Browser}
\label{subsec:unterstuetzterfunktionsumfanggaengigerbrowser}
Um genauere Aussagen über den unterstützten Funktionsumfang von mobilen Endgeräten
und Desktop-PCs zu bekommen, wurden die Desktop-Browser von ihrem mobilen
Gegenstück getrennt betrachtet. Zu der Recherche des Funktionsumfangs wird
der von Thomas Steiner aus der wissenschaftlichen Arbeit "What is
in a Web View? An Analysis of Progressive Web App Features When the
Means of Web Access is not a Web Browser" bereitgestellte
"PWA Feature Detector"\footnote{https://github.com/tomayac/pwa-feature-detector/} sowie 
die Webseite "Can I use"\footnote{https://caniuse.com/} verwendet.
Der "PWA Feature Detector" ermittelt anhand von im HTML-DOM
vorhandenen Feldern, ob der Browser bestimmte Merkmale unterstützt.
Dabei werden neben den Feldern \code{window}, \code{document} und \code{navigator}
auch Felder im Service Worker erkannt. Die Resultate der Analyse können von dem
eigentlich unterstützten Funktionsumfang abweichen. Die von Thomas
Steiner für relevant angesehenen Merkmale sind Folgende:
Offline Capabilities; hierbei wird nach
einer Unterstützung der Cache API gesucht. Push Notifications;
ist die Push API unterstützt? Add to Home Screen; ist es möglich, PWAs auf
dem Startbildschirm zu speichern (A2HS)? Background Sync; ermöglicht eine eventgetriebene Synchronisation
zwischen dem Service Worker und der Anwendung. Navigation Preload; die Möglichkeit
Ressourcen bereits während dem Hochfahren des SW vorzuladen
(Navigation Preload Manager API). Silent Push; das Starten von durch das
Backend ausgelösten Prozessen im Frontend, ohne den Benutzer
durch eine Nachricht davon zu informieren (Web Budget API).
Storage Estimation; die Nachfrage nach genutztem Speicher sowie
dessen Beschränkung (Storage Manager). Persistent Storage;
das Speichern von Daten, die nur mit Anfrage an den Benutzer gelöscht
werden dürfen (Storage Manager). Web Share; das Teilen von Inhalten,
über die durch die Plattform bereitgestellten Operationen (Web Share API).
Media Session; das Bereitstellen von Inhalten für native
Bereiche wie den Lockscreen (Media Session API). Media Capabilities;
die Möglichkeit das perfekte Format von Inhalten anhand von
Informationen, die vom Browser bereitgestellt werden, auszuliefern (Media Capabilities API).
Device Memory; wie viel RAM stehen dem Betriebssystem zur
Verfügung (Device Memory API)? Detect Native App; ist die native Version
der Anwendung bereits installiert? Payment Request;
die Benutzung von durch das Betriebssystem bereitgestellten
Zahlungsverfahren (Payment Request API). Credential Management;
dem Benutzer die Verwendung von im Betriebssystem gehorteten
Anmeldedaten zu ermöglichen (Credential Management API).\cite[Abschnitt 2.3]{WhatIsInAWebView}

Bei der Analyse des unterstützten Funktionsumfangs wurden die neuesten Versionen\footnote{Stand Januar 2020 (Internet Explorer 11, Edge 79, Firefox 72, Chrome 79, Safari 13, Opera 64, Safari iOS 13.2, Opera Mini 46, Android Browser Chromium 76, Opera Mobile 46, Chrome für Android 79, Firefox für Android 68, UC Browser für Android 12.12 und Samsung Internet 10.1) }
relevanter Browser\footnote{Relevant für die Arbeit anhand der in \Cref{subsec:gestaltung} (2. Absatz) aufgeführten Statistiken.}
begutachtet. Die Ergebnisse würden in zwei Tabellen aufgeteilt. In \Cref{tab:unterstutzterpwafunktionsumfangteil1}
ist der unterstützte Funktionsumfang der Desktop-Browser dargestellt. Der unterstützte Funktionsumfang der Browser
für mobile Endgeräte ist in \Cref{tab:unterstutzterpwafunktionsumfangteil2} aufgelistet.

Zwei der fünfzehn in \Cref{subsec:markmale} erwähnten Merkmale wurden aufgeteilt.
So gibt es zusätzlich zur Background Sync noch eine
periodische Background Sync. Die Periodic Background Sync API, eine Erweiterung der Background
Sync API, ist momentan noch in der Entwicklung und wird vorraussichtlich erst mit
Chrome 80 veröffentlicht. Die API wird das Ausführen von Hintergrundaufgaben der SW
in einem periodischen Interval ermöglichen.\cite{ChromeStatusPeriodicBackgroundSync}
Die Web Share API ist in zwei Level unterteilt. In der Web Share API Level 1 war es
nur möglich, Texte und Uniform Resource Locators (URLs) zu teilen. Mit der Web Share
API Level 2 können nun auch Dateien geteilt werden.\cite{WebShareAPITwoChromeStatus}
Zusätzlich zur Payment Request API wurde die Payment Handler API hinzugefügt. Mit dieser
API ist es möglich, Zahlungsdaten, Lieferanschriften und Kontaktdaten, in einer durch
den Service Worker gesteuerten Bedienungsoberfläche abzufragen.\cite{PaymentHandlerAPIGoogleDev}

\begin{table}[h]
\begin{center}
\scalebox{0.8}{
\begin{tabular}{l*{6}{c}}
PWA Merkmale & IE & Edge & Firefox & Chrome & Safari & Opera \\
\hline
Offline Capabilities    &  -  & \cm & \cm & \cm & \cm & \cm \\
Push Notifications	    &  -  & \cm & \cm & \cm &  -  & \cm \\
Add to Home Screen	    &  -  &  -  &  -  & \cm &  -  & \cm \\
Background Sync	        &  -  &  -  &  -  & \cm &  -  & \cm \\
Periodic Bg. Sync       &  -  &  -  &  -  &  -  &  -  &  -  \\
Background Fetch        &  -  &  -  &  -  & \cm &  -  & \cm \\
Navigation Preload	    &  -  & \cm &  -  & \cm &  -  & \cm \\
Storage Estimation	    &  -  &  -  & \cm & \cm &  -  & \cm \\
Persistent Storage	    &  -  &  -  & \cm & \cm &  -  & \cm \\
Web Share Level 1	    &  -  &  -  &  -  &  -  & \cm &  -  \\
Web Share Level 2       &  -  &  -  &  -  &  -  &  -  &  -  \\
Media Session           &  -  &  -  &  -  & \cm &  -  & \cm \\
Media Capabilities	    &  -  &  -  & \cm & \cm & \cm & \cm \\
Device Memory	        &  -  &  -  &  -  & \cm &  -  & \cm \\
Detect Related App      &  -  &  -  &  -  & \cm &  -  &  -  \\
Payment Request	        &  -  & \cm &  -  & \cm & \cm & \cm \\
Payment Handler	        &  -  &  -  &  -  & \cm &  -  & \cm \\
Credential Mgmt	        &  -  &  -  &  -  & \cm &  -  & \cm \\
\end{tabular}
}
\end{center}
\caption{Unterstützter PWA-Funktionsumfang für Desktop-PCs}
\label{tab:unterstutzterpwafunktionsumfangteil1}
\end{table}

\begin{table}[h]
\resizebox{\textwidth}{!}{%
\begin{tabular}{l*{8}{c}}
PWA Merkmale & \specell{iOS\\Safari} & \specell{Opera\\Mini} & \specell{Android\\Browser} & \specell{Opera\\Mobile}  & \specell{Chrome\\Android} & \specell{Firefox\\Android} & \specell{UC\\Browser} & \specell{Samsung\\Internet} \\
\hline
Offline Capabilities   & \cm & \cm & \cm & \cm & \cm & \cm & \cm & \cm \\
Push Notifications	   &  -  &  -  &  -  & \cm & \cm & \cm & \cm & \cm \\
Add to Home Screen	   & \cm & \cm & \cm & \cm & \cm & \cm & \cm & \cm \\
Background Sync	       &  -  & \cm & \cm & \cm & \cm &  -  & \cm & \cm \\
Periodic Bg. Sync      &  -  &  -  &  -  &  -  &  -  &  -  &  -  &  -  \\
Background Fetch       &  -  &  -  &  -  & \cm & \cm &  -  &  -  &  -  \\
Navigation Preload	   &  -  & \cm & \cm & \cm & \cm &  -  &  -  & \cm \\
Storage Estimation	   &  -  & \cm & \cm & \cm & \cm &  -  & \cm & \cm \\
Persistent Storage	   &  -  & \cm & \cm & \cm & \cm &  -  & \cm & \cm \\
Web Share (Level 1)	   & \cm &  -  &  -  & \cm & \cm &  -  &  -  & \cm \\
Web Share (Level 2)    &  -  &  -  &  -  &  -  & \cm &  -  &  -  &  -  \\
Media Session          &  -  &  -  &  -  & \cm & \cm &  -  &  -  & \cm \\
Media Capabilities	   & \cm & \cm & \cm & \cm & \cm & \cm &  -  & \cm \\
Device Memory	       &  -  & \cm & \cm & \cm & \cm &  -  &  -  & \cm \\
Detect Native App      &  -  &  -  &  -  &  -  & \cm &  -  &  -  &  -  \\
Payment Request	       & \cm &  -  &  -  &  -  & \cm &  -  &  -  & \cm \\
Payment Handler	       &  -  & \cm & \cm & \cm & \cm &  -  &  -  & \cm \\
Credential Mgmt	       &  -  & \cm & \cm & \cm & \cm &  -  &  -  &  -  \\
\end{tabular}
}
\caption{Unterstützter PWA-Funktionsumfang für mobile Endgeräte}
\label{tab:unterstutzterpwafunktionsumfangteil2}
\end{table}

Analysiert man die Resultate von \Cref{tab:unterstutzterpwafunktionsumfangteil1} und
\Cref{tab:unterstutzterpwafunktionsumfangteil2}, wird klar ersichtlich, dass Chrome
die PWA Entwicklung vorantreibt. Es überrascht auch kaum, dass der Internet Explorer,
bei dem die Entwicklung im März 2015 eingestellt wurde, \cite{HeiseInternetExplorer} keine der PWA-Merkmale
unterstützt. Die A2HS-Funktionalität wird von allen
mobilen, in dieser Auswertung betrachteten Browsern unterstützt.
Bei den mobilen Browsern ist der unterstütze Funktionsumfang insgesamt größer als
bei den Browsern für Desktop-PCs. Ein Grund dafür ist, dass man sich bei den Merkmalen der progressiven Webanwendungen
hauptsächlich an der Optik und Haptik von nativen Apps orientiert. Chrome ist aktuell
der einzige Browser, bei dem man eine PWA auch auf einem Desktop-PC installieren
kann. Grundfunktionalitäten, wie  die Möglichkeit der Offlineverwendung, werden fast von allen
Browsern unterstützt. Die wichtigste Erkenntnis dieser Auswertung ist Folgende: Die einzelnen
PWA-Merkmale müssen als Anreicherung betrachtet werden. Es ist nicht davon auszugehen, dass
der Browser des Benutzers ein bestimmtes Merkmal unterstützt.

Um den unterstützten Funktionsumfang besser beurteilen zu können, ist es wichtig,
sich den Browser-Marktanteil der eigenen Zielgruppe zu vergegenwärtigen.
So verwenden laut StatCounter weltweit mehr als 50\% der Endbenutzer Google Chrome und besitzten somit
alle PWA-Merkmale, die Chrome bereitstellt.\cite{StatCounterBrowserMarketShare}

\subsection{Trusted Web Activities}
\label{subsec:trustedwebactivities}
Das größte Defizit einer PWA ist, dass diese nicht im App Store gefunden werden kann.
Für viele Nutzer ist es ungewohnt, PWAs direkt aus dem Browser heraus auf ihr mobiles Endgerät zu installieren.
Seit Google Chrome Version 72 bietet Google hierfür eine Lösung an.
Mit sogenannten Trusted Web Activities (TWA) ist es möglich, PWAs auch in den App Store
zu bringen. Eine TWA ist eine Methode, den Inhalt einer PWA in eine Android App\footnote{Dies ist mithilfe von Google Chromes Custom Tabs möglich.}
zu integrieren.\cite{TrustedWebActivitiesGoogle}

Ein Verweis auf ein im Webserver liegendes Zertifikat stellt sicher,
dass die Webseite und die App von dem gleichen Entwickler
stammen. Ein Vorteil, gegenüber dem Einbinden einer PWA in einen WebView, ist es,
dass die TWA und somit die native App nicht kontinuierlich aktualisiert werden muss,
um den Browser auf dem neuesten Stand zu halten. Bei einem WebView ist die Browser-Version in
die Anwendung integriert. Eine TWA hingegen beinhaltet keinen eigenen Browser, sondern verwendet
den Standardbrowser des Betriebssystems.

Trusted Web Activities werden mit der Verwendung von \code{<activity>} Tags in das Android-App-Manifest integriert. Die
Ausführung der Anwendung im Browser bringt Vor- und Nachteile mit sich.
Ein Vorteil ist, dass die TWA Zugriff auf im Browser gespeicherte Daten wie den LocalStorage
hat. Ein schwerwiegender Nachteil ist allerdings, dass der installierte Browser des mobilen Endgerätes
die Verwendung der TWA unterstützen muss. Ist dies nicht der Fall, wird die Anwendung in einem sichtbaren
Browser-Tab angezeigt. Andreas Domin erwähnt in einem Blogbeitrag auf Digital Pioneers eine Möglichkeit,
dies zu umgehen: Hierfür "[...] wird nativer Java- oder Kotlin-Code benötigt, um zu überprüfen, ob Chrome
(oder ein Browser, der Custom Tab unterstützt) installiert ist.
Ist dem nicht so, kann eine alternative Technologie geladen werden:
eine Webview zum Beispiel."\cite[2. Abschnitt]{TrustedWebActivitiesT3N}
Trusted Web Activities werden aktuell nur von Google und somit Android unterstützt. Um die PWA
im App Store von Apple zu veröffentlichen, muss ohnehin eine WebView verwendet werden.

\section{Microservices}
\label{sec:microservices}
Dieser Abschnitt setzt sich mit dem Architekturmuster "Microservices" auseinander.
In \Cref{subsec:wasverstehtmanuntermicroservices} stellt sich die Frage, was Microservices
eigentlich sind und was sie auszeichnet. Die Vor- und Nachteile einer Microservice-Infrastruktur
werden in \Cref{subsec:vorundnachteile} diskutiert.

\subsection{Was versteht man unter Microservices?}
\label{subsec:wasverstehtmanuntermicroservices}
Der Begriff "Microservices" ist derzeit sehr populär.\footnote{Seit Februar 2014 ist das Interesse
an dem Begriff "Microservices" auf Google Trends konstant angestiegen.\cite{MicroservicesGoogleTrends}}
Wenn der Begriff "Microservices" zum ersten Mal gehört wird, denkt man womöglich an das Architekturmuster
"Serviceorientierte Architektur" (SOA). Der Begriff SOA wurde bereits 1996
von Gartner, einem Unternehmen, das Forschung in der IT-Branche
betreibt\footnote{https://www.gartner.com/en/about/}, verwendet \cite{GartnerSOAPart1}.
SOA ist ein Erklärungsmodell zur Aufteilung von Geschäftslogik in einzelne Services \cite{SOAManifesto}.

Was sind Microservices und was ist der Unterschied zu SOA?
In dem Buch "Microservices" von Eberhard Wolff werden Microservices als Ansatz zur Modularisierung
von Software definiert. Den Unterschied zu vorhandenen Architekturmustern beschreibt Wolff wie folgt:
"Das Neue: Microservices nutzen als Module einzelne
Programme, die als eigene Prozesse laufen."\cite[S. 2]{MicroservicesBook}
Durch neue Technologien wie Docker, eine Open-Source Software
zur Kapselung von Prozessen in umgebungsunabhängige Instanzen, \cite{DockerOverview}
ist es möglich, eine serviceorientierte Architektur flexibler
zu gestalten. Softwarelösungen wie Docker stellen eine einheitliche Umgebung in
allen Prozessstufen der stetigen Softwareauslieferung bereit.
Ob in einer lokalen Entwicklungsumgebung oder beispielsweise in einem
Kubernetes-Cluster, die Umgebung und somit das
Verhalten der Anwendung ist gleich.\footnote{Von Einflussfaktoren
wie veränderbaren Umgebungsvariablen und externen Schnittstellen abgesehen.}

Das Wort "Microservice" beinhaltet den Wortteil "Micro",
bekannt als Einheitenvorsatz für Millionstel. Die Komplexität eines Softwareprojektes
wird oftmals in Lines of Code (LOC) gemessen. Ausgehend
von Diskussionen auf StackExchange siedelt sich ein komplexes Softwareprojekt bei
ca. 2 Millionen LOC an.\cite{ProjectsizeStackexchange}
Daraus würde folgen, dass ein sogenannter "Microservice" eine Größe um etwa
2 Code-Zeilen haben sollte. Diese Annahme ist von realen
Anwendungsfällen weit entfernt. Eberhard Wolff sieht den Ansatz,
die Größe eines Microservices anhand von Code-Zeilen zu
messen, als kritisch an. Code-Zeilen seien von der verwendeten
Technologie abhängig. Außerdem sollte man sich bei der
Serviceuntergliederung lieber nach der "...fachlichen Domäne richten".\cite[S. 31 und 32]{MicroservicesBook}

\subsection{Vor- und Nachteile}
\label{subsec:vorundnachteile}
In \Cref{subsec:wasverstehtmanuntermicroservices} werden Microservices
als einzelne, in einer Containervirtualisierung laufende Programme beschrieben. Sind diese Programme
zustandslos, lassen sie sich einfach und effizient skalieren. Ein zustandsloser Service hat die Eigenschaft, dass zwei
aufeinanderfolgende Anfragen nicht an die gleiche Instanz des Programms gesendet werden müssen.
Beinhaltet der Service eine Datenbank, benötigt es eine weitere Strategie der Skalierung
wie Database Sharding\footnote{Ein Verfahren zur horizontalen Partitionierung großer Datenbanken in kleinere Bestandteile.\cite{DatabaseShardingMicrosoft}}
oder Multi-Master Replication\footnote{Die in der Arbeit verwendete PostgreSQL bietet
Bi-Directional Replication (BDR), ein asynchrones Multi-Master Replikationssystem, an.\cite{PostgresMultiMasterReplication}}.
Durch die Lastverteilung auf eigenständige Services
wird der Schwellwert, ab welchem man horizontal skalieren muss, erst später erreicht.\footnote{Im Vergleich zu einem Monolith.}
Eberhard Wolff kritisiert den Performanzverlust, der durch die
Kommunikation über die verteilten Systeme entsteht. So schreibt er wie folgt: 

\begin{quote}
"Aufrufe durch das Netzwerk sind deutlich langsamer als lokale Aufrufe.
Aus reiner Performanz-Perspektive kann es sinnvoller sein, mehrere Microservices
zusammenzulegen oder Technologien zu nutzen, die auf lokale Aufrufe setzen."\cite[S. 65, 3. Absatz]{MicroservicesBook}
\end{quote}

Um die Kommunikation zwischen den Services zu beschleunigen, werden daher Message Broker wie
Apacke Kafta oder RabbitMQ verwendet. \mbox{Nichtsdestotrotz} kann die Performanz einer solchen
Kommunikation im Regelfall nicht mit der internen Kommunikation eines einzelnen Programms Schritt halten.

In dem Artikel "Warum es nicht immer Microservices sein müssen" auf Informatik Aktuell
sprechen Christoph Iserlohn und Till Schulte-Coerne von einer Externalisierung der Komplexität.
Um komplexe Querzugriffe innerhalb des eigenen Systems zu vermeiden, würden Teams oftmals 
zu einer Microservice-Umgebung wechseln. "Allerdings kaufen wir uns dies teuer ein.
Wir verstecken durch die Trennung nämlich eine nicht unerhebliche Komplexität in
der Kommunikation zwischen den Systemen und machen diese quasi unsichtbar."\cite[Paragraph 11]{InformatikAktuellWarumNichtImmerMicroservices}

Microservices sind keineswegs als Wundermittel gegen eine verbesserungswürdige Architektur eines
Monolithen zu sehen. Bei der Entscheidung für eine Microservice-Infrastruktur geht man, wie
auch bei allen anderen Architekturmustern, einen Kompromiss ein. So gewinnt man deutlich
an Flexibilität, bezahlt dies aber mit einem Anstieg an Komplexität des Gesamtsystems.

\section{Kontinuierliche Integration und Auslieferung}
\label{sec:kontinuierlicheintegrationundauslieferung}
Continuous Delivery (CD) ist das Verfahren, Software in einem stetigen Zyklus auszuliefern.
Neue Technologien, wie die Containervirtualisierungssoftware Docker,
die im März 2013 von dotCloud ins Leben gerufen wurde \cite{DockerAbout2014}, vereinfachen
den stetigen Softwareauslieferungsprozess enorm. So ist es möglich, in einem automatisierten
Auslieferungsprozess, Schnappschüsse der Container (Images) zu erstellen, diese zu testen und selbige nach erfolgreichem
Bestehen der Tests in der Produktion zu verwenden. Eberhard Wolff schreibt über das Ziel von Docker in
seinem Buch "Continuous Delivery" Folgendes:

\begin{quote}
"Im Vergleich zu normalen Prozessen trennen Linux-Container
einzelne Anwendungen viel besser voneinander, wenn es darum geht, die Software auf einem anderen
System zu installieren. Docker bietet ein Repository für Images an, so dass eine Vielzahl
von Servern mit identischen Images betrieben werden können."\cite[S. 56]{ContinuousDeliveryWolff}
\end{quote}

Somit kann sichergestellt werden, dass die Software und deren Umgebung in der Testphase
sowie in der Produktion dieselben sind.

Immer mehr Anbieter stellen Werkzeuge für Continuous Integration (CI) zur Verfügung,
die es ermöglichen eine CI/CD-Pipeline anhand von Konfigurationsdateien aufzuziehen. So bieten
neben Travis CI, CircleCI und Jenkins nun auch Versionsverwaltungshosts, wie Github und
Gitlab, CI-Werkzeuge an.\footnote{Github bietet seit dem 13. November 2019 eigene CI/CD-Werkzeuge an.\cite{GithubCIToolsHeise}}
Ein Vorteil eines CI/CD-Ansatzes ist eine schnelle Feature-Auslieferung und somit eine schnellere Resonanz der Benutzer.
Außerdem wirkt sich das stetige Ausführen automatisierter Tests in der Pipeline positiv auf die Robustheit der Software aus.
Zwei weitere Vorteile sind das einfache Reproduzieren von Fehlern sowie das unkomplizierte Zurückrollen der ausgelieferten Version.
Ein Nachteil eines CI/CD-Ansatzes ist der große Aufwand des initialen Aufbaus der automatisierten Prozesse.
Bei komplexen Änderungen des Systems müssen zudem große Teile des automatisierten Prozesses angepasst werden.
