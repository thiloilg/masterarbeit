\chapter{Die Qualitätssicherung}

\section{Die Dokumentation}

\section{Die testgetriebene Entwicklung}

Finster war's, der Mond schien helle auf die grünbeschneite Flur, als
ein Wagen blitzesschnelle langsam um die runde Ecke fuhr. Drinnen
saßen stehend Leute schweigend ins Gespräch vertieft, als ein
totgeschossner Hase auf dem Wasser Schlittschuh lief und ein
blondgelockter Knabe mit kohlrabenschwarzem Haar auf die grüne Bank
sich setzte, die gelb angestrichen war.
Finster war's, der Mond schien helle auf die grünbeschneite Flur, als
ein Wagen blitzesschnelle langsam um die runde Ecke fuhr. Drinnen
saßen stehend Leute schweigend ins Gespräch vertieft, als ein
totgeschossner Hase auf dem Wasser Schlittschuh lief und ein
blondgelockter Knabe mit kohlrabenschwarzem Haar auf die grüne Bank
sich setzte, die gelb angestrichen war.

\section{CI, CD und nochmal CD}
\label{sec:ci-cd-und-nochmal-cd}


Finster war's, der Mond schien helle auf die grünbeschneite Flur, als
ein Wagen blitzesschnelle langsam um die runde Ecke fuhr. Drinnen
saßen stehend Leute schweigend ins Gespräch vertieft, als ein
totgeschossner Hase auf dem Wasser Schlittschuh lief und ein
blondgelockter Knabe mit kohlrabenschwarzem Haar auf die grüne Bank
sich setzte, die gelb angestrichen war.
Finster war's, der Mond schien helle auf die grünbeschneite Flur, als
ein Wagen blitzesschnelle langsam um die runde Ecke fuhr. Drinnen
saßen stehend Leute schweigend ins Gespräch vertieft, als ein
totgeschossner Hase auf dem Wasser Schlittschuh lief und ein
blondgelockter Knabe mit kohlrabenschwarzem Haar auf die grüne Bank
sich setzte, die gelb angestrichen war.


\subsection{Continuous Delivery}
\label{subsec:continuous-delivery}
Wenn Sie bei der Abkürzung CD immernoch an eine Compact Disc denken, muss ich Sie leider enttäuschen. Jene ist ein veralteter optischer Speicher, welcher heutzutage nur noch in Computermuseen oder Rückdeckeln diverser wissenschaftlicher Arbeiten zu finden ist. Aber darum geht es hier garnicht. Was ich damit meine, ist Continuous Delivery, zu deutsch stetige Softwareauslieferung.


\section{Die Zentrierung service-relevanter Einstellungen}
% extrahierung von theme colors etc.

\section{Was für Tests gibt es, und wenn ja, wieviele?}

\section{Usability Tests mit realen Nutzern}
7 Dialogprinzipien achten!
\section{SCRUM im Alleingang}

\section{Zwing dich zu deinem Glück}

\subsection{Die Einrichtung von Githooks}
\section{Testing Stages}

\section{Die Projektstruktur}

\section{Die Arbeitsablaufoptimierung}

\section{Die standige Reduzierung von Abhängigkeiten}

\section{Das allseits beliebte Refactoring}

\section{Wann ist Performance wichtig?}

\section{MVP}

\section{Wartung der CI Tests}
%% automatisiertes clean up von container images

\section{Die Sicherheit}
\subsection{JWT}
\subsection{SSO}
\subsection{Two-factor Auth}
\subsection{Two-Way Validation}
\subsection{Die Rechteverwaltung}
\subsection{Whitelist statt Blacklist}
\subsection{One verification route}