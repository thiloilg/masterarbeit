\chapter{Fazit}
\label{chap:fazit}

In der Arbeit wurde ein Gesamtsystem zur Visualisierung von Daten konzipiert und
umgesetzt. Bei der Gestaltung der Softwarearchitektur wurde sehr viel Wert auf
die Flexibilität der Anwendung gelegt. Durch die Trennung der einzelnen Komponenten
wie die Datenverarbeitung, die Benutzerverwaltung und die Visualisierung der Daten
in eigene Komponenten, ist es möglich, diese flexibel zu erweitern,
getrennt zu skalieren und wenn nötig auszutauschen. 

Die Verwendung des Plugin-Ansatzes sind der Visualisierung der Daten keine Grenzen gesetzt.
So können neue Ansammlungen an Diagrammen mit komplett unabhängigen Frameworks und
Bibliotheken entwickelt werden. Durch die Auslieferung der Daten über
das WebSocket-Protokoll, können diese in Echtzeit im Frontend aktualisiert werden.
Durch die Verwendung der Technologien, die eine progressive Webanwendung mit sich bringt,
wurde es ermöglicht, eine Anwendung für Desktop-PCs, Smartphones und Tablets aus einer einzigen
Codebasis heraus zu entwickeln. Die Frontendanwendung wurde außerdem in dem Android App Store
veröffentlicht und weißt native Charakteristiken wie die Navigationsbar am unteren
Ende des Smartphone-Bildschirms auf. 

Caching-Strategien und Kompressionsalgorithmen steigern die Performanz des Gesamtsystems.
Zudem wird das System in einem stetigen Auslieferungsprozess aktualisiert, ohne das
der Benutzer eigenhändig ein Update installieren muss. Dies gilt auch für die Android App.
Durch eine aktivgeführte Dokumentation der Schnittstellen der einzelnen Services
ist es möglich, diese ohne Einlesen in den Quellcode zu verwenden. Eine solide Abdeckung der Integrationstests sichert überdies die
Fehlerfreie Kommunikation der einzelnen Komponenten. Das Gesamtsystem wurde auf bis zu
fünf Servern ausgeführt und bewiesen, dass das Gesamtsystem durchaus auch in der Produktion
verwendet werden kann.

All diese Einscheidungen über die Softwarearchitektur und die verwendeten Technologien
formen eine Gesamtanwendung, die nicht nur robust und zuverlässig ist, sondern auch
als solides, gutdurchdachtes Fundament für eine zukünftige Softwarelösung
verwendet werden kann.
