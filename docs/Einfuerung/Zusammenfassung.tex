\chapter*{Zusammenfassung}
\addcontentsline{toc}{chapter}{Zusammenfassung}
\label{chap:zusammenfassung}

Um die Effizienz eines Unternehmens zu steigern, benötigt dieses zeitaktuelle Informationen über seine Arbeitsabläufe.
Zwischen der Datenerfassung und dem Aha-Erlebnis, das beim Blick auf ein aufschlussreiches Dashboard entsteht, liegt ein
langer Weg. Glücklicherweise findet die Arbeit in Zusammenarbeit mit der Firma GBO Datacomp statt, die für die
Datenerhebung, Verdichtung und Speicherung bereits fertige Lösungen anbietet. Die Aufgabe der Masterarbeit
ist es, ein System zu schaffen, das die Daten performant und sicher transportiert, flexibel verarbeitet und dem Benutzer
in einem webbasierten Dashboardingtool zugänglich macht.

Die Entwicklung einer solchen Gesamtanwendung ist eine echte Herausforderung. Von der Infrastruktur
über die Softwarearchitektur bis hin zur Gestaltung der einzelnen Komponenten ist ein stetiges
Verfahren abverlangt, das plant, realisiert, untersucht und verbessert.

Zur Umsetzung des Systems wurde die Logik der Gesamtanwendung in einzelne Microservices
unterteilt. Zur Verwaltung der Benutzerdaten bietet die Arbeit einen Node.js Server
mit einer REST-Schnittstelle an. Die zu visualisierenden Daten werden in einem
in Go geschriebenen Dienst verarbeitet und über das WebSocket-Protokoll an die
Frontendanwendung gesendet. Das Frontend besteht aus einer progressiven Webanwendung,
die ein agiles Verfahren zur Erstellung der Dashboards bereitstellt. Die zur Darstellung
verwendeten Diagramme werden während der Laufzeit aus einem Redis-Cache zugeliefert.

Das System wurde erfolgreich implementiert und getestet. Die Anwendung ist unter der Domain \code{blicc.org}
sowie im Google Play Store erhältlich.
