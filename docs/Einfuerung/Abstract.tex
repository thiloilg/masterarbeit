\chapter*{Abstract}
\addcontentsline{toc}{chapter}{Abstract}
\label{chap:abstract}

To increase the efficiency of a company, it requires realtime
information about its workflows. Between the data capturing
and the aha experience, which you get by viewing a 
insightful dashboard, lies a long journey. Fortunately,
the master thesis is collaborating with the company gbo datacomp GmbH,
which already provides solutions for the assessment, compaction and
storage of the data. The thesis deals with the design of a system,
which transports data in a performant and secure way, provides
flexible processing and gives users the opportunity to access
the data in a web-based dashboarding tool.

The development of such a complex system is a major challenge.
From the infrastructure via the software architecture through to
the design of a single component, a continuous procedure which includes
planing, implementation, testing and improvement is required.

The overall system has been divided into several microservices. A Node.js
server provides a REST-API to manage user data. A service written in Golang is
responsible for the data processing. It uses the WebSocket protocol to stream
processed data to the frontend application. The frontend exists of a progressive
web app which provides an agile way to create dashboards. Diagrams used to visualize
data are served over a Redis cache during runtime.

The system has been successfully implemented and tested. The application is available
at \code{blicc.org} as well as on the google play store.
