\chapter{Die Ziegel der zielgetriebenen Entwicklung}
\label{chap:die-ziegel-der-zielgetriebenen-entwicklung}
Was sind die Ziegel, also die Grundbestandteile der zielgetriebenen Entwicklung? Zuerst einmal kann man die zielgetriebene Entwicklung in zwei Hauptbestandteile untergliedern. Die Zieldefinition und die Sicherstellung der Einhaltung des Ziels. Bei der Zieldefinition ist es wichtig, die Sollwerte, also die zu erbringenden Teilziele so zu definieren, dass es 
\section{Wieso testgetriebene Entwicklung?}
\label{sec:wieso-testgetriebene-entwicklung}

\section{Was versteht man unter CD?}
\label{sec:was-versteht-man-unter-cd}
Wenn Sie bei der Abkürzung CD immernoch an eine Compact Disc denken, muss ich Sie leider enttäuschen. Jene ist ein veralteter optischer Speicher, welcher heutzutage nur noch in Computermuseen oder Rückdeckeln diverser wissenschaftlicher Arbeiten zu finden ist. Aber darum geht es hier garnicht. Was ich damit meine, ist Continuous Delivery, zu deutsch stetige Softwareauslieferung.
