\chapter*{Zusammenfassung}
\addcontentsline{toc}{chapter}{Zusammenfassung}
\label{chap:zusammenfassung}

Im Zuge der digitalen Revolution des 20. Jahrhunderts hat sich die
Menge der erfassten Daten auf der ganzen Welt massiv gesteigert. In
komplexen IoT-Systemen werden täglich millionen von Datenpunkten
erfasst und gespeichert. Die Daten alleine haben allerdings keinen
allzu großen Mehrwert. Erst durch die Erkenntnisse der Auswertung
können zukünftige Prozesse effizienter gestaltet werden. Um diesen
Vorgang zu optimieren, benötigt es eines performanten, zuverlässigen 
sowie sicheren Systems, welches mit großen Datenmengen umgehen kann
und diese so intuitiv wie möglich dem Nutzer in Form eines agilen
Dashboardingtool zur Auswertung bereitstellt.

Diese Arbeit stellt die Softwarearchitektur sowie die Entwicklung
eines solchen Systems in den Mittelpunkt. Bereits während dem
Entstehungsprozess wird dabei auf eine Anzahl von Qualitätsmerkmalen
geachtet. So wird die Änderbarkeit durch eine wohl durchdachte Architektur
mit austauschbaren Komponenten, die Robustheit durch eine immer
wiederkehrende Reihe an automatisierten Tests, die Sicherheit durch bewährte
Verifizierungs- und Validierungsverfahren und die Wartbarkeit durch
gut dokumentierte Schnittstellen und leserlichen Quellcode gewährleistet.
