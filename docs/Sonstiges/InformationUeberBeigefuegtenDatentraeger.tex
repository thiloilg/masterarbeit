\chapter{Information über beigefügten Datenträger}
\label{chap:informationueberbeigefuegtendatentraeger}
Im Rückdeckel dieser Arbeit befindet sich eine SD-Karte. Diese beinhaltet
die Masterarbeit in digitaler Form sowie den Quellcode der Gesamtanwendung.

\begin{small}
\begin{itemize}[noitemsep,nolistsep]
    \item \textbf{Masterarbeit als PDF}
    \item \textbf{Quellcode}
    \begin{itemize}[noitemsep,nolistsep]
        \item \codeBold{/.vscode} VS Code Einstellungen für Debugging und Ordnersymbole
        \item \codeBold{/android} Werkzeuge zur Verpackung der PWA in eine Android App
        \item \codeBold{/api} Repository des "Resource Management Services"
        \item \codeBold{/app} Repository der progressiven Webanwendung
        \item \codeBold{/certs} Ablage der Zertifikate für die lokale Entwicklung
        \item \codeBold{/db} PostgreSQL als Datenbank des "Resource Management Services"
        \item \codeBold{/delivery} Repository des "Data Delivery Services"
        \item \codeBold{/docs} Dokumentation der Gesamtanwendung
        \item \codeBold{/mock-api} API-Mock für die Integrationstests
        \item \codeBold{/mongo} MongoDB zur Speicherung der zu visualisierenden Daten
        \item \codeBold{/plugins} Repository zur Einwicklung der Diagramme
        \item \codeBold{/redis} In-Memory-Datenbank zur Auslieferung der Plugins
        \item \codeBold{/scripts} Shell-Skripte zum Aufsetzen der Infrastruktur
    \end{itemize}
\end{itemize}
\end{small}

\bigskip

\begin{small}
\begin{myboxi}[Hinweis]
Die benötigten Schritte zum Aufsetzen der Gesamtanwendung sind in der 
\code{README.md} Datei im Wurzelverzeichnis des Quellcode-Ordners beschrieben.
\end{myboxi}
\end{small}
