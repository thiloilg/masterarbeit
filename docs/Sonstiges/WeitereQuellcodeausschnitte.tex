\chapter{Weitere Quellcodeausschnitte}
\label{chap:weiterequellcodeausschnitte}

\section*{Web-App-Manifest}
\label{sec:webappmanifest}
Dieser Quellcodeausschnitt zeigt das \code{manifest.json}, welches in dem Wurzelverzeichnis
des Webservers einer progressiven Webanwendung liegen muss. Aus Platzgründen wurde im Quellcodeausschnitt
\ref{lst:webappmanifest} nur ein Icon aufgelistet.

\begin{listing}[h]
    \inputminted{jsx}{snippets/json/manifest.json}
    \caption{Web-App-Manifest}
    \label{lst:webappmanifest}
\end{listing}

\newpage

\section*{Resource Management Service Routen}
\label{sec:resourcemanagementservicerouten}
Die Tabelle \ref{tab:resourcemanagementservicerouten} zeigt die
Routen der REST-API des "Resource Management Services" in Zusammenhang
mit den HTTP-Methoden. Dabei steht grundsätzlich \code{GET} für den Zugriff,
\code{POST} für das Erstellen, \code{PUT} für das Aktualisieren und \code{DELETE}
für das Löschen einer Ressource. Ausnahmen sind hierbei beispielsweise Routen wie
\code{/users/:id/delete}, bei welchen eine zusätzliche Übertragung der Anmeldedaten
gefordert wird. Für die Routen \code{/charts}, \code{/dashboards} und \code{data-sources}
sind Query-Parameter wie \code{fields=id,title}, \code{search=oee}, \code{skip=10} und
\code{take=2} bereitgestellt. Somit kann die Anzahl der aufgelisteten Ressourcen
sowie deren Felder eingeschränkt werden.

\begin{table}[h]
\begin{center}
\begin{tabular}{llc}
Route & HTTP-Methods \\
\hline
/bundles/:slug          & [GET, PUT]         \\
/charts                 & [GET, POST]        \\
/charts/:id             & [GET, PUT, DELETE] \\
/dashboards             & [GET, POST]        \\
/dashboards/:id         & [GET, PUT, DELETE] \\
/data-sources           & [GET, POST]        \\
/data-sources/:id       & [GET, PUT, DELETE] \\
/health-check           & [GET]              \\
/health-check/auth      & [GET]              \\
/refresh                & [POST]             \\
/tokens                 & [POST, DELETE]     \\
/two-factor-auth        & [GET, POST]        \\
/two-factor-auth/delete & [POST]             \\
/users                  & [GET, POST]        \\
/users/:id              & [GET, PUT]         \\
/users/:id/delete       & [POST]             \\
\end{tabular}
\end{center}
\caption{Resource Management Service Routen}
\label{tab:resourcemanagementservicerouten}
\end{table}

\newpage

\section*{Data Delivery Service Ein- und Ausgangsdaten}
\label{sec:datadeliveryserviceeinundausgangsdaten}

Die folgenden zwei Quellcodeausschnitte \ref{lst:datadelvieryrequestbodybeispiel}
und \ref{lst:datadeliveryresponsebodybeispiel} zeigen die Ein- und
Ausgangsdaten einer Anfrage gegen den "Data Delivery Service".

\begin{listing}[h]
    \inputminted{jsx}{snippets/json/data-delivery-example/request.data-delivery.txt}
    \caption{Data Delivery Request-Body Beispiel}
    \label{lst:datadelvieryrequestbodybeispiel}
\end{listing}

\begin{listing}[h]
    \inputminted{jsx}{snippets/json/data-delivery-example/response.data-delivery.json}
    \caption{Data Delivery Response-Body Beispiel}
    \label{lst:datadeliveryresponsebodybeispiel}
\end{listing}

\newpage

\section*{JSON-Datei zur Speicherung eines Dashboards}
\label{sec:jsondateizurspeicherungeinesdashboards}

Der Quellcodeausschnitt \ref{lst:jsondateizurspeicherungeinesdashboards}
zeigt die JSON-Datei, die alle Informationen beinhaltet, die zum
automatisierten Erstellungsprozess eines Dashboards benötigt werden.
Unter \code{arrangement} sieht man die Anordnung und unter \code{settings}
die Einstellungen eines Dashboards.

\begin{listing}[h]
    \inputminted{jsx}{snippets/json/dashboard-arrangement/anordnung.json}
    \caption{JSON-Datei zur Speicherung eines Dashboards}
    \label{lst:jsondateizurspeicherungeinesdashboards}
\end{listing}
