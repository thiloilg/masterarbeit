\chapter{Weitere Quellcodeausschnitte}
\label{chap:weiterequellcodeausschnitte}

\section*{Web-App-Manifest}
\label{sec:webappmanifest}
Dieser Quellcodeausschnitt zeigt das \code{manifest.json}, welcher in dem Wurzelverzeichnis
des Webservers einer progressiven Webanwendung liegen muss. Aus Platzgründen wurde nur ein
Icon aufgelistet.
\begin{listing}[h]
    \inputminted{jsx}{snippets/json/manifest.json}
    \caption{Web-App-Manifest}
    \label{lst:webappmanifest}
\end{listing}

\newpage

\section*{Resource Management API Routes}
\label{sec:resourcemanagementapiroutes}

\begin{table}[h]
\begin{center}
\begin{tabular}{llc}
Route & HTTP-Methods \\
\hline
/bundles/:slug          &  [GET, PUT]         \\
/charts                 &  [GET, POST]        \\
/charts/:id             &  [GET, PUT, DELETE] \\
/dashboards             &  [GET, POST]        \\
/dashboards/:id         &  [GET, PUT, DELETE] \\
/dashboards             &  [GET, POST]        \\
/dashboards/:id         &  [GET, PUT, DELETE] \\
/data-sources           &  [GET, POST]        \\
/data-sources/:id       &  [GET, PUT, DELETE] \\
/health-check           &  [GET]              \\
/health-check/auth      &  [GET]              \\
/tokens                 &  [POST, DELETE]     \\
/two-factor-auth        &  [GET, POST]        \\
/two-factor-auth/delete &  [POST]             \\
/users                  &  [GET, POST]        \\
/users/:id              &  [GET, PUT]         \\
/users/:id/delete       &  [POST]             \\
\end{tabular}
\end{center}
\caption{Resource Management API Routes}
\label{tab:unterstutzterpwafunktionsumfangteil1}
\end{table}
